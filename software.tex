\chapter{Software}
\label{ch:software}

\section{\github}

Our code has transitioned from bazaar to \git. A more powerful svn with
advantages ... The source code of our software has been open sourced
through \github. Easier access, better collaboration.

Conforming to coding styles, PEP8.

Documentation through \github Pages.


\section{\hisparc Public Database}

\subsection{\api}

Access to metadata

Functions have been added to both \sapphire and \jsparc to access the
\api. These take care of constructing the urls, converting the json to a
sensible format for the the given language.


\subsection{Event Summary Database}

Derived database with certain analyses applied, as explained later. Also
coincidences already found, next; automated reconstructions


\section{\jsparc}

Javascript library.


\subsection{Data retrieval tool}

Data analysis in any modern browser (IE9+, Firefox, Chrome, Safari)
Interpolation between different datasets. Can even make histograms.

\section{\sapphire}


\subsection{\pypi}

\sapphire has been made available for easy installation via the Python
Package Index (\pypi).


\subsection{Clusters}

Using the \api \sapphire has access to the \gps coordinates of all
stations. And for those stations that have submitted their detector
positions, those as well. This makes it easy to setup a cluster with any
selection of \hisparc stations. This can be used for simulations.
Imaginary stations can be added to plan future station locations.

\subsection{\corsika}

Reading \corsika output, convert to \hdf and use for simulations.


\subsection{Refactored simulations}

The simulation section of \sapphire has been completely rewritten
working together with dr. David Fokkema and Hans Montanus. Works with
\corsika data, more consistent output that can eb analysed the same way
as real \hisparc data (except that the input is also known).

