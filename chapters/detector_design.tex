\section{Detector design}

The scintillator detector consists of two main components which perform the
detection. The scintillator, which emits light as charged particles pass
through it, and the Photomultiplier tube which converts the scintillation
photons into an electric signal.

The scintillator is a cuboid of \SI[product-units=power]{100 x 50 x
2}{\centi\meter} made from BC-408 \cite{bc408}. The BC-408 material was
chosen for its good timing properties and light output for charged particles.

The Photomultiplier tube (PMT) (\cite{et:pmt}, also ref hamamatsu) is a small
(?) diameter PMT that matches with the thickness of the scintillator.

Besides these the detector construction consists of several other components.
An isosceles triangle light guide, which is glued to the scintillator using
Optical Cement (\cite{bc600}). A little adapter piece is attached to the square
end of the triangle with Optical Cement and the round window of the \pmt with
optical tape. This construction is wrapped in thin, \SI{30}{\micro\meter},
aluminium foil with patches of thicker, ??, aluminium foil at the corners of
the scintillator, to prevent the sharp edges from piercing the thin foil. This
is then wrapped in thick, ??, light tight black pond foil to keep light out and
protect the aluminium foil from outside influences.


\subsection{Photo multiplier}

The PMT is used to convert the optical signal from the scintillator to an analog/electrical signal. The PMT must amplify the signal to be much higher than the noise level in the readout electronics. A PMT 

n photons
QE
gain
gain vs time

An alternative for PMTs might be Silicon based sensors (pixels/chips). However, silicon pixels these are not as thermally stable as PMTs. The temperature in a \hisparc skibox can fluctuate over \SI{30}{\celcius} in a day. The voltage on such (avalanche) chips needs to be controlled carefully to keep them performing optimally. Multi-pixel photon counters (MPPC) have been tested and compared to PMTs \cite[Chapter~3.4]{lio2011} and \cite[Chapter~3.6]{lio2010}. The MPPC have a smaller collection area, smaller temperature range (require cooling?), and increased complexity when using multiple (for increased area/reduced dark counts) MPPC. The conclusion is that at that time the MPPC are not yet ready for use in \hisparc.

The PMTs are attached to the detector with optical tape, allowing them to be replaced easily.

Nikhef designed high-voltage voltage-divier circuit.
Cockcroft-Walton voltage multiplier circuit.
High linearity?

