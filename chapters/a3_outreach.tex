\chapter{Outreach}
\label{ch:outreach}

\section{Leraar in Onderzoek}

The Leraar in Onderzoek (LiO) programme gives high-school physics
teachers the opportunity to participate in current research. Each year
several participants works at the \nikhef one day per week for the
entire year. A first they will be introduced into the relevant topics
for the research, then a project plan is determined for each person. At
the end of the year each will produce an article detailing their
research and results, these are bundled into one report
\cite{lio2009, lio2010, lio2011, lio2012}.


\section{Infopakket}

In late 2013 a new effort was started to provide Dutch high-school
physics teachers with easy to use course materials including assignments
and practical assignments.

Testing by various interested teachers in 2015.


\section{Data access}

All data for the \hisparc stations is freely available to everyone.
Using \sapphire the raw data can be acquired. However, this requires
some proficiency with \python. The Public Database is meant to ease data
access.


\subsection{Download data forms}

With the addition of the ESD to the Public Database data access has become easier. A web based download form allows data to be downloaded as \tsv. This lowers the threshold for data access from requiring \python skills to just opening a \tsv file in Excel for example. Not only \hisparc event data is offered for download, also data from \hisparc weather stations and \knmi lightning detections can be downloaded.


\subsection{Coincidences}

To improve insight in the contributions of individual stations to the
\hisparc network we have added pages that show the number of
coincidences between all \hisparc stations in a day. If more stations
are online, more coincidence will occur. Unfortunately this is partly
due to accidental coincidences, two unrelated events. However, when
stations are relatively close together the likelihood increases that the
events in the coincidences are from the same air shower. The time window
that is used to define when events are in coincidence has to be
carefully chosen to prevent to many accidental coincidences. This data
can also be downloaded as \tsv.


\section{\jsparc \javascript library}

To provide more tools for students a \javascript library has been
developed that can access and interpret the Public Database API and the
data download. Several websites have been developed around this library
to provide an interface to the data access.


\subsection{Data retrieval tool}

Data analysis in any modern browser.
Interpolation between different datasets. Can even make histograms and fits.

Recognizes all \tsv files from the Public Database.


\subsection{Other tools}

Other tools include; API interface to explore the options of the API,
Trigger simulation to simulate random triggers, and Station distances to
easily find the distance between any two stations.
