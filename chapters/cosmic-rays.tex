\chapter{Cosmic rays}
\label{ch:cosmic-rays}


\section{A brief history of cosmic-ray research}

% Observed effects of cosmic rays without special instruments, e.g. aurora.

The effects of low-energy solar cosmic rays were already seen in ancient times. This was in the form of the aurora borealis and aurora australis. Several ancient written recordings of the phenomenon \cite{stephenson2004aurora} have been found, moreover, many folk tales about this phenomenon exist. The cause of the aurora's was not known then. In 1600 Gilbert created a terrela, a sphere with a magnetic field resembling that of the Earth \cite{gilbert1893terrela}. In 1896 Birkeland \cite{birkeland1896aurora} showed that electron beams fired at a terrela were bent along the field lines to form circular regions around the magnetic poles like the aurora. The aurora appears to be caused by disturbances in the magnetosphere of the Earth. The disturbances are generally caused by solar winds. Magnetic reconnection occurs and causes charged particles to be accelerated into the atmosphere \cite{angelopoulos2008reconnection}. There the particles will excite and ionize oxygen and nitrogen atoms. When these de-excite or recombine the auroral colors are produced. An example of the aurora can be seen in \cref{fig:aurora}.

\begin{figure}
    \centering
    \includegraphics[width=0.6\textwidth]{plots/cosmic-rays/aurora.png}
    \caption{Aurora borealis as seen in Rovanemi, Finland on 17 March 2015, St. Patricks day. The large aurora activity on that day was due to a G4 geomagnetic storm on the sun two days earlier. The interactions between the incoming electrons and the atoms in the atmosphere creates the fluorescence. In this case the red and green colors are caused by different excitation energies in the atomic oxygen.}
    \label{fig:aurora}
\end{figure}

% Briefly talk about the discovery of the background radiation (caused by cosmic rays) and the discovery of where they come from (i.e. not radiation from Earth).
% How it was discovered that cosmic rays were charged particles.
% How this all lead to cosmic-ray research and the discovery of extensive air showers.

\subsection{Discovery of cosmic rays}

In 1785 the effects of cosmic rays were detected by de Coulomb \cite{coulomb1785electroscope} [alternative, in english] with an electroscope. An electroscope measures the amount of electrical charge in a test object. De Coulomb saw it unexpectedly discharge in the air, even though it was properly insulated. In 1895 the discovery of X-rays and their capability to ionize air \cite{roentgen1895radiation,flakus1981radiation} presented a good candidate for why the electroscope discharged. However, a source of the X-rays had not yet been identified. In 1896 Becquerel discovered the existence of radioactivity. he also found that the radiation from radioactive material could cause an electroscope to discharge. Further experiments with the electroscopes were performed to assertain the source of the ionizing radiation. Experiments went over seas, underwater \cite{pacini2011sea}, underground, and far above ground (Eiffel Tower [Wulf1909]), but no conclusive evidence for cosmic rays was found until Hess performed balloon flights in 1911 and 1912 which accurately determined the amount of radiation at various altitudes. It showed a clear increase at very high altitudes, indicating an extra-terrestial source for the radiation. Clay [Clay1927] found that the intensity of the ionization depended on latitude. This latitudinal dependency could be caused by the magnatic field of the Earth. Being affected by magnetic fields indicates that the particles were charged particles. The indepent discovery of extensive air showers (EAS) in the atmosphere by Auger (1939) \cite{auger1939eas} and Rossi (1934) ignited the cosmic-ray research field.


\section{Cosmic ray source and their journey}

Cosmic rays are very energetic particles moving through space. Some impact Earth's atmosphere causing cascades of particles. Most commonly cosmic rays are protons, heavier atomic nuclei are the next most common component. A small fraction of the cosmic ray flux are electrons and gamma rays. Anti-particle cosmic rays have also been observed.

% What (some of) the expected origins/sources of cosmic-rays are.
The cosmic rays reaching Earth have energies anywhere from \SIrange{e9}{e20}{\eV}. Objects in space where these particles are accelerated have not yet been unequivocally identified. A strong contributing suspect are Supernova Remnants (SNR). A SNR consists of shockwaves of particles ejected by supernovae. A shockwave occurs when particles are moving faster than the local speed of sound. SNRs pass through the Interstellar Medium (ISM), the region between stars in the galaxy, picking up particles and possibly accelerating them. The Fermi Large Area Telescope (LAT) discovered evidence for \Pgpz-decay in SNRs \cite{ackermann2013snr}, which is an indicator for energetic protons, because neutral pions can be created in proton-proton collisions. Models predict that SNRs can contribute a significant fraction of the cosmic-ray flux up to \SI{e15}{\eV} [CaprioliICRC2015]. Models predicting the creation of the very high energetic cosmic rays (VHECR) in Active Galactic Nuclei (AGN) and pulsars exist, but evidence for these processes has not yet been discovered.

% The acceleration mechanisms that come into play.
[Fermi, CaprioliICRC2015]

\subsection{Cosmic voyage}

% What the cosmic rays encounter along the way.
Cosmic rays do not travel through a perfect vacuum. The ISM is filled with particles, radiation and magnetic fields. The interstellar matter consists mainly of atomic hydrogen. With an average density of \SI{1}{particle\per\centi\meter\cubed} or \SI{e-23}{\gram\per\centi\meter\cubed}. Imagine a proton travelling the diameter of the Milky Way (\SI{40}{\kilo\parsec}, or \SI{1.2e23}{\cm}) before reaching Earth. This particle will have passed through a column depth of \SI{1.2}{\gram\square\centi\meter}. This is very small relative to the thickness of Earth's atmosphere.

% Explantation for the GZK cutoff/limit.
Radiation fields can in some cases affect the cosmic rays. With energies above \SI{5e19}{\eV} proton cosmic rays can interact with photons from the cosmic microwave background (CMB). In these collisions pions can be produced (via \PDelta) and the primary cosmic ray may loose up to \SI{20}{\percent} of its energy. The density of the CMB photons is \SI{410}{\per\centi\meter\cubed}. With a cross section of ... For a \SI{e20}{\eV} proton this gives a mean free path of \SI{8}{\mega\parsec}. This upper limit of cosmic-rays is known as the Greisen Zatsepin Kuzmin (GZK) limit [ref to GZK]. If a source of extremely energetic cosmic rays is close enough they may make it without undergoing these energy loss interactions.

% How the direction of cosmic rays no longer points to their origin, gyro radius.
Cosmic rays do not simply travel in a straight line from their source to Earth. Moving charged particles are affects by magnetic fields. The Lorentz force may change the direction of the particle. The strength of this depends on the angle between the direction of motion and magnetic field lines. [Lorentz force].
The strength of all magnetic fields in the universe is not known. The overal field strength and distribution of random fields for the Milky Way are being determined from experiments. The average magnetic field strengths are \SI{~.3}{\nano\tesla} [grigat2011, fix number add fluctuations]. The radius of curvature for relativistic particles in magnetic fields is called the Larmor radius (or gyro radius). This is given by

\begin{equation}
    R = \SI{108}{\kilo\parsec}
        \frac{E_{\si{\exa\eV}}}{Z B_{\si{\nano\tesla}}}
\end{equation}

where $R$ is the radius of curvature, $Z$ the atomic number of the particle, which is equal to the charge because atomic cosmic rays are fully ionised, $B$ is the magnetic field strength, and $E$ the energy of the particle [jansson2010magnetic]. In \cref{fig:gyroradius} the gyroradius of a proton cosmic ray is shown for various energies of the cosmic ray and various strengths of the magnetic field.

\begin{figure}
    \centering
    \includegraphics[width=0.6\textwidth]
                    {plots/cosmic-rays/gyroradius}
    \caption{[Gyro radius versus energy and magnetic field]
The gyro radius of primary proton cosmic rays as a function of their energy for different magnetic field strengths. The amount of deflection greatly depends on the actual average magnetic field strength in the Milky Way, which is not precisely known. Indicated are sizes of important astronomical objects.}
    \label{fig:gyroradius}
\end{figure}

% This leads to isotropy at low energies, and possibly anisotropy at very high energies.
If the energy of a cosmic ray is low enough it may be captured by a galaxy. This can happen if the gyro radius is significantly smaller than the object. For the Milky Way this transition is around \SIrange{e16}{e18}{\eV}. For particles at and below these energies the paths though the ISM will not be straight. The deflections will cause the directions of the particles at Earth to be randomised, i.e. isotropically distributed. A low enough energies the particles are also affected by the solar wind and then anisotropy may occur. Above these energies, the particle paths will be straighter. Unless it encounters strong local fluctuations in the magnetic field. For cosmic rays of high enough energy anisotropy may be expected if there are clearly defined sources on the sky.

Heavy particles have smaller gyro radii and are therefore more easily contained. Consequently, heavy particles may be accelerated to higher energies at their source, being contained until accelerated to high enough energies to escape. The cosmic rays of energies above this transition can escape from the Milky Way. Similarly cosmic rays in other galaxies can escape if they have enough energy. From observations the average magnetic field strength in other visible galaxies is on average \SI{.4}{\nano\tesla} [jansson2010magnetic]. In the intergalactic medium (IGM) the particle density is approximately \SI{1}{particle\per\meter\cubed}. The expected average magnetic field in the IGM is \SI{.003}{\nano\tesla} [get correct number].

Knowledge about the exact distribution, strength and change over time of the magnetic fields in the universe are not at out disposal. This makes it difficult/impossible to fully reconstruct the path of cosmic rays.

\subsection{Final destination}

% Explain what cosmic rays are seen at Earth, as observed by experiments.
% How various models explain the spectrum; different compositions, galactic, extra-galactic.
At Earth cosmic rays arrive from all directions. On average the flux as a function of different energies is constant. In \cref{fig:spectrum} the measured differential flux of cosmic rays at the top of the atmosphere is shown. At low energies the flux varies over time due to solar modulation. The solar cycle of \SI{11}{\year} affects the flux in this energy region of cosmic rays. The top axis shows the equivalent center of mass collision energies. This can be used to relate the energy in the cosmic ray collision to the collision energies reported for particle colliders on Earth. At the 'end' of the spectrum, around \SI{e20}{\eV} the spectrum steepens and seems to end. No cosmic rays above $E = \SI{e21}{\eV}$ have yet been detected. The reasons for this is probably the GZK limit explained earlier. But may also be explained by the lack of sources being able to produce such highly energetic particles.

\begin{figure}
    \centering
    \includegraphics[width=0.6\textwidth]{plots/cosmic-rays/spectrum.pdf}
    \caption{Differential flux of primary cosmic rays versus the kinetic energy per particle. The spectral index of the spectrum is approximately $\gamma = 2.7$. Around $E = \SI{e15}{\eV}$ is the transition from directly (lower energies) to indirectly (higher energies) detected cosmic rays. At low energies the flux is indicated by a band because it changes over time due to solar modulation. Less low energy cosmic rays are observed during high solar activity. [Todo:] Equivalent center of mass collision energies as top x-axis. Marked are the achieved p-p (and p-Pb) collision energies in colliders. The transition from Galactic and extra-Galactic is determined from the expected gyro radii in the Milky Way. Also shown are likely source contributions at some energies (SNR).}
    \label{fig:spectrum}
\end{figure}


% Explain the possibilities of direct detection of cosmic rays: Low energy, balloon, space.
Cosmic rays can be detected in multiple ways. Either by detecting them directly or indirectly. In order to directly detect cosmic rays the detector needs to be above or high in the Earth's atmosphere (above \SI{25}{\kilo\meter}). This is typically done by balloon-borne experiments or space-based missions. Due to the low flux of high energy cosmic rays these methods are mainly aimed at the lower energies ($E < \SI{e15}{\eV}$). For cosmic rays of higher energies the particle cascades they create in the atmosphere are used. These cascades can be detected from the ground. The physics and characteristics of these particle showers are discussed further in \cref{sec:cr:eas}.

\subsection{Direct detection}

% What is learned from direct detection experiments: Composition, Solar modulation, van Allen belt, heliosphere, particle-antiparticle fraction, isotropy.
Being able to directly detect the particle has the advantage of easy particle identification. The main downside is the flux of cosmic-rays decreases steeply when looking at higher-energy cosmic rays. Ideally the detector has a large detection area and a very long exposure time. Unfortunately a detector that has to be flown to the top of the atmosphere or in space can only be so big before it costs too much to launch. Balloon experiments are often short runs where the balloon carrying the detector stays up for about a week. The JACEE \cite{asakimori1998jacee} and RUNJOB \cite{hareyama2011runjob} experiments flew multiple times. Each experiment reached a total fly time of \SI{60}{\day}. Notable space-based missions are PAMELA \cite{adriani2014pamela}, in orbit on the Resurs DK1 sattelite since 2006, and the AMS-02 \cite{casaus2014ams} experiment attached to the ISS since 2011. These space-based experiments have been operating for many years. The measurements provide good resolution on the composition of cosmic rays at energies upto the knee \cite{kulikov1958knee}.

These experiments are able to measure the modulation of the low-energy cosmic-ray flux due to the solar activity. During high solar activity (many sunspots) the flux of galactic cosmic rays at Earth is reduced \cite{adriani2013modulation}. This is the reason for the band seen at low flux in \cref{fig:spectrum}. Also the ratio of normal and anti-particles has be determined [PAMELA/AMS/BESS ICRC]. For electrons about \SI{15}{\percent} of the particles around \SI{e11}{\eV} are positrons. \SI{2e-4}{\percent}. No anisotropy has been found in the arrival directions of e+e- cosmic rays detected by these experiments [PAMELA ICRC]

Together these direct detection experiments provide a very detailed picture of the low energy cosmic-ray spectrum. The per-nucleus spectrum of such experiments is shown in \cref{fig:low_e_spectrum}.


\begin{figure}
    \centering
    \includegraphics[width=0.6\textwidth]
                    {plots/cosmic-rays/PDG_28_1_fluxes_per_nucleus.pdf}
    \caption{Decomposed differential flux of primary cosmic rays. Balloon and space experiments directly detecting cosmic-rays provide composition data for the cosmic-rays.}
    \label{fig:PDG_28_1_fluxes_per_nucleus}
\end{figure}

The longest running space-based cosmic-ray experiments are the Voyager spacecrafts. The Pioneer probes, also carrying a cosmic-ray detector, were launched about 4 years earlier, but contact with them was lost over a decade ago. The Voyager spacecrafts were launched in 1977 to use the planetary alignment to accelerate with several gravity assists [ref Stone, ICRC]. In August 2011 Voyager I seems to have reached the heliopause of the solar system, at \SI{121}{\astronomicalunit} distance to the Sun. Here the detection rate of particles caused by the solar wind (>\SI{5e8}{\eV}) decreased steeply and the rate of cosmic-ray particles from outside the solar system (>\SI{7e10}{\eV}) increased significantly (see figure/ref). Cosmic rays of these energies are thought to originate from supernovae in this galaxy.

\begin{figure}
    \centering
    \includegraphics[width=0.6\textwidth]
                    {plots/cosmic-rays/voyager_heliosphere}
    \caption{Extrasolar cosmic ray rate by Voyager 1, stone2015]
Timeline of the cosmic ray rate measured by the Voyager 1 spaceprobe. The rate started to increase when it passed the termination shock. After passing the heliopause the rate has stabilized at 4 times the rate inside the heliosphere. This is a measure of the cosmic ray rate unaffected by solar winds.}
    \label{fig:voyager_heliosphere}
\end{figure}


\subsection{Indirect detection}

% How high-energy cosmic rays may hold information about anisotropies and possible sources (hot spot).

Currently the two biggest ground-based cosmic-ray experiments are the Pierre Auger Observatory (Auger) in Argentina and the Telescope Array (TA) in Utah. Both use particle detectors and fluorescence detectors overlooking the particle detector area. Auger has been expanded with radio detectors. Because of their size (\SI{3000}{\square\kilo\meter} and \SI{700}{\square\kilo\meter} respectively). and separation between the particle detectors (\SI{1.5}{\kilo\meter} and \SI{1.2}{\kilo\meter}) these experiments are looking for cosmic-rays at the highest energies. The minimum energy threshold for both experiments, ignoring the smaller low energy extensions, is above \SI{e18}{\eV}. Several cosmic-rays of $E > \SI{e20}{\eV}$ have been detected by both epxeriments, settings limits on the high-energy cosmic ray spectrum and probing the GZK limit. At these high energies possibilities for anisotropy exist. In \cref{fig:skymap_ta_auger} a combined sky map of cosmic rays detected by both experiments shows a dipole moment deviations from isotropy. When considering only primaries with energies above $\SI{57e18}{\eV}$ the TA sees a slight preference for cosmic rays from a region on the sky, shown in \cref{fig:hotspot_ta}. However, with a significance of $3.4 \sigma$ more data is needed to verify the excess.

\begin{figure}
    \centering
    \includegraphics[width=0.6\textwidth]
                    {plots/cosmic-rays/skymap_ta_auger}
    \caption{[verzi2015] Anisotropy measurement for cosmic rays of $E > \SI{e19}{\eV}$ by the combined Pierre Auger Observatory and Telescope Array data. The overlapping regions are used to correct for sensitivity differences between the experiments. Only a dipole moment is seen, higher multipoles do not deviate from the expected fluctuations of an isotropic flux at \SI{99}{\percent} confidence level.}
    \label{fig:skymap_ta_auger}
\end{figure}

\begin{figure}
    \centering
    \includegraphics[width=0.6\textwidth]
                    {plots/cosmic-rays/hotspot_ta}
    \caption{Hot spot by TA on mollweide plot, abasi2015]
Anisotropy significance plot for the Telescope Array data for primary cosmic rays of $E > \SI{57e18}{\eV}$. Here a hot spot with a significance of $3.4 \sigma$ is observed. This may indicate a source, but more data is required for a more definite conclusion.}
    \label{fig:hotspot_ta}
\end{figure}

% Highlight some of the features of the high energy spectrum, and that there is some composition information known (light-heavy) but no exact fractions.
In \cref{fig:PDG_28_8_all_particle_spectrum} the height of the spectrum is compressed by scaling the data by $E^{2.6}$. This way certain features are more easily seen. For instance the changes in the steepness of the power law (knees and ancles) are more visible. Composition measurements are less specific for indirect measurements. They do provide an indication for the overall lightness (mostly protons) or heaviness (more biased towards iron) of the composition. It appears that beyond $E = \SI{e18}{\eV}$ the composition becomes heavier. However, the model predictions upon which these measurements depend are not yet perfect. Improvements to the experiments are planned to increase the exposure and improve the sensitivity to the composition.

\begin{figure}
    \centering
    \includegraphics[width=0.6\textwidth]
                    {plots/cosmic-rays/PDG_28_8_all_particle_spectrum}
    \caption{Detailed spectrum at high energies]
The differential flux of cosmic rays multiplied by $E^{2.6}$. This representation of the latest data from cosmic-ray experiments reveals more structure in the spectrum. Little data is available for the highest energy cosmic rays, but limits are set on the maximum rates. The composition of cosmic rays at these energies is not yet known.}
    \label{fig:PDG_28_8_all_particle_spectrum}
\end{figure}


\section{Extensive air showers}
\label{sec:cr:eas}

% Typical hadronic interactions of (hadronic) cosmic rays entering the atmosphere.
% The cross section of such interactions, as known from colliders and cosmic ray experiments.
% Colliders not yet upto highest energies, and different type of interactions.
The first interaction target of incoming cosmic rays are the atoms in the upper atmosphere. The upper atmosphere consists mainly of nitrogen (\SI{78.09}{\percent}), oxygen (\SI{20.95}{\percent}), and argon (\SI{0.93}{\percent}) [fix composition for upper atmosphere]. Unfortunately collider experiments rarely use these nuclei as collision material. So the exact cross section of interactions between cosmic-rays (from protons to iron) and atomic nitrogen, oxygen, and argon is not known at high energies. However, from p-p collisions the p-Air cross sections can be extrapolated using Glauber theory. The LHC also performs p-Pb and Pb-Pb collisions..


- Provide the model of the thickness of the atmosphere as a function of the height (as used in CORSIKA).
- The combination of cross section and atmospheric depth gives a distribution for the likely first interaction altitudes.
- Explain that these hadronic interactions can produce many secondaries (multiplicty).
- Explain what particles are produced and what happens to them.
- Explain how energy is transferred into the leptonic part of the shower.
- Explain Heitler model for EM shower.
- This results in a longitudinal profile.
- Particle creation on one side with energy transfer from hadronic to EM
- Particle decay and absorption cause reduction in number of shower particles.
- The interactions also result in deflections, so the shower expands lateraly.
- Shower shower profile (CORSIKA).
- Each shower with the same primary energy can develop very differently, chance processes. The size (number of particles) varies for cosmic rays of same energy.


%\begin{equation}
%\HepProcess{\Pproton + \Pproton \to magic \to \Pproton + \Pproton +
%\Ppizero},
%\end{equation}
%
%\begin{equation}
%%\begin{split}
%\HepProcess{\Ppiplus &\to \APmuon + \Pnum}, \\
%\HepProcess{\Ppiminus &\to \Pmuon + \APnum}, \\
%\HepProcess{\Ppizero &\to \Pphoton + \Pphoton}.
%%\end{split}
%\end{equation}
%
%\begin{equation}
%%\begin{split}
%\HepProcess{\Pmuon &\to \Pelectron + \APnue}, \\
%\HepProcess{\APmuon &\to \Ppositron + \Pnue}.
%%\end{split}
%\end{equation}

\begin{figure}
    \centering
    \includegraphics[width=0.6\textwidth]
                    {plots/cosmic-rays/pair_crosssection}
    \caption{Cross section p-Air vs energy, grieder2010]
The measured proton-air cross section determined from cosmic ray measurements of the first interaction altitude. As the energy of the particles increases, so does the cross section. The measurements are within the predicted values from proton-proton cross sections measured in colliders when corrected with Glauber theory.}
    \label{fig:pair_crosssection}
\end{figure}


\begin{figure}
    \centering
    \includegraphics[width=0.6\textwidth]
                    {plots/cosmic-rays/atmospheric_depth}
    \caption{Height vs atmospheric depth, corsika74000]
The atmospheric depth as a function of altitude above ground for typical atmospheric conditions in the Netherlands (CORSIKA model?). The air becomes thinner higher above the ground.}
    \label{fig:atmospheric_depth}
\end{figure}

\begin{figure}
    \centering
    \includegraphics[width=0.6\textwidth]
                    {plots/cosmic-rays/first_interaction_altitude}
    \caption{Distribution of first interaction altitude for showers of certain energy (penetration depth), corsika74000]
Distribution of first interaction altitudes for primary proton cosmic-rays of $E = \SI{16}{\eV}$. From the cross sections a mean free path can be calculated for cosmic rays entering the atmosphere.}
    \label{fig:first_interaction_altitude}
\end{figure}


\begin{figure}
    \centering
    \includegraphics[width=0.6\textwidth]
                    {plots/cosmic-rays/multiplicity}
    \caption{Multiplicity in hard hadron interaction versus energy, grieder2010]
Average secondary particle multiplicty in p-p collisions as a function of the collision energy. The multiplicty increases with the energy of the collision. Results from collider data is shown along with models that try to predict the behavior at higher energies. Predictions vary widely from model to model.}
    \label{fig:multiplicity}
\end{figure}

\begin{figure}
    \centering
    \includegraphics[width=0.6\textwidth]
                    {plots/cosmic-rays/schematic_shower}
    \caption{Schematic shower representation (perhaps separate for hadronic/em), engel2011]
Simple representation of the hadronic and electromagnetic cascades in an air shower. The high multiplicity causes many new charged and neutral particles to be created. The subsequent hadronic interactions are at lower energies and thus, on average, lower multiplicty. After ~6 interaction lengths the shower has reached ground level and most energy will have gone to the electromagnetic shower. For the electromagnetic cascade start when neutral pions or muons decay into gammas or electrons. Gamma's undergo pair creation and electrons (and positrons) emit gammas due to bremsstrahlung. At each radiation length the number of particles doubled, as long as enough energy is available.}
    \label{fig:schematic_shower}
\end{figure}


\begin{figure}
    \centering
    \includegraphics[width=0.6\textwidth]
                    {plots/cosmic-rays/longitudinal_profile}
    \caption{Longitudinal shower profile, with contributions by various particles, engel2011]
The longitudinal shower profile showing the number of particles in the air shower as a function of the atmospheric depth. Different types of particles are shown separately. The depth at which the maximum number of particles exists is called Xmax.}
    \label{fig:longitudinal_profile}
\end{figure}

\begin{figure}
    \centering
    \includegraphics[width=0.6\textwidth]
                    {plots/cosmic-rays/shower.png}
    \caption{CORSIKA shower simulation to show spacial development, corsika74000]
Particle tracks of a simulated air shower from a proton with $E = \SI{e16}{\eV}$. The different types of particles are shown with different colors. This shows both the longitudinal and lateral distribution of particles.}
    \label{fig:shower}
\end{figure}

\begin{figure}
    \centering
    \includegraphics[width=0.6\textwidth]
                    {plots/cosmic-rays/shower_size_distribution}
    \caption{Shower size distributions for showers of various energies, corsika74000]
Number of leptons on ground distributions versus the energy of the primary particle. The variation from shower to shower means that showers will have a different number of particles reaching ground level, eventhough the primary energy was equal. The variations can be as high as a factor 10.}
    \label{fig:shower_size_distribution}
\end{figure}


\section{Shower front}

- What the shower looks like as it reaches ground, how this shape is a result of the various interactions.
- The important distributions related to the shower front.
- LDF for various components, and what influences it.
- At various core distances the temporal profile.
- The energy distribution of the particles in the front (see if they are high/low/close to decay).
- Explain why muons are earlier than electron/gamma.
- Explain why the front is curved and the thickness of the front varies.
- The relation between the curvature/rise time to other shower properties.

\begin{figure}
    \centering
    \includegraphics[width=0.6\textwidth]
                    {plots/cosmic-rays/schematic_front}
    \caption{schematic representation of shower front]
A schematic representation of the shower front as it reached ground level. The dots represent shower particles. This is a snapshot in time, so all particles are moving with approximately $c$ to the ground. The particles on the front will reach the ground first followed by those behind. Near the core the density of particles in higher and they are closer to the front. Away from the core the front is curved away from a flat shower front plane. The thickness (rise time) of the front increases with core distance.}
    \label{fig:schematic_front}
\end{figure}

\,

\begin{figure}
    \centering
    \includegraphics[width=0.6\textwidth]
                    {plots/cosmic-rays/ldf}
    \caption{Lateral density distribution]
The particle density as a function of core distance for a shower of $E = \SI{e16}{\eV}$. The contribution from te different components is shown.}
    \label{fig:ldf}
\end{figure}

[Lateral energy distribution, average energy per particle versus core distance]
The (mean particle energy or energy density) as a function of core distance. Close to the code the particles have more energy?

\begin{figure}
    \centering
    \includegraphics[width=0.6\textwidth]
                    {plots/cosmic-rays/temporal_profile}
    \caption{Temporal structure of shower front, for various core distances]
The temporal structure of the shower front at different core distances for different types of particles. $t = \SI{0}{\ns}$ is the time the first particle reaches ground. Muons arrive first, followed by the gammas and electrons. Each has a long tail of stragglers.}
    \label{fig:temporal_profile}
\end{figure}

\,

\begin{figure}
    \centering
    \includegraphics[width=0.6\textwidth]
                    {plots/cosmic-rays/curvature_front}
    \caption{Front curvature as function of size (or interaction altitude)]
The mean shower front curvature as a function of the shower size. The interaction altitude of a shower affects the curvature of its shower front.}
    \label{fig:curvature_front}
\end{figure}

[Front rise time as function of size (or interaction altitude)]
The mean shower front rise time at various core distances as a function of the shower size. The combined rise time for muons and electrons is shown.
