\chapter{Cosmic rays}
\label{ch:cosmic-rays}

Cosmic rays are highly energetic particles from extraterrestrial sources.
This can be any kind of particle.


\section{Current state}

Possible anisotropies at very high energies. High energy spectrum
getting more data, but not completely in agreement (due to hotspots,
models or calibrations?). Work is being done to better identify the
contributions to the spectrum from the different kinds of particles.
Very model dependent analyses.


\section{Experiments}

There are several very large experiments currently in operation that are
working on the frontier of cosmic-ray research.

The Pierre Auger Observatory (http://www.auger.org)
\cite{auger:prototype}, located in Argentina, is a large ground-based
detector using multiple detection methods in conjunction. Cherenkov
water tanks are used to detect charged particles (leptons) as they pass
through the detector. Fluorescence telescopes are used to look for
ultra-violet light in the atmosphere. Fluorescence occurs in atmospheric
nitrogen which is exited by the extensive air shower. Finally radio
antennae are used to detect the radio signal produced when the
positively and negatively charged particles in the air shower are
separated by the magnetic field. A similar project exists in the
Northern hemisphere called the Telescope Array.

IceCube (http://icecube.wisc.edu), located in Antarctica, is a neutrino
telescope. Neutrinos pass through most matter because of their low
interaction cross sections. In order to detect the neutrinos IceCube
uses Earth as a detector. IceCube uses \pmts buried \SI{2}{\kilo\meter}
deep in the ice of Antarctica to detect the Cherenkov light caused by
muons that are created by neutrinos passing through the Earth.
Additionally a part called IceTop sits on top of the ice ... detect
atmospheric muons etc.. A similar project called \kmnet is currently
being deployed/build in the Northern hemisphere.

Large area detectors are required to detect the rare very-high-energy
cosmic rays. In figure (spectrum) the flux of primary cosmic rays of
specific energies can be seen.

The Auger Observatory (and others?) has been operating for over
\SI{10}{\year}. Cosmic rays with energies exceeding
\SI{57}{\exa\electronvolt} have been detected. Why do they use 57 EeV?
Sources/escape velocity/random/models/detector/hot spot?

Challenges
Spectrum
GZK
GZ

Energy reconstruction
CR Sources
Populations


\section{Air showers}

Cosmic rays induce extensive air showers when they enter interact with particles in the atmosphere.


\subsection{Energy/density distribution in front}

Number of detected events correlates to atmospheric pressure.
Air shower grows and dies out due to interaction of the particles in the air shower. When there is
... LDF


\subsection{Front thickness and curvature}

The shower front is often approximated as a flat surface  of particles
perpendicular to the shower axis moving at nearly the speed of light
along the shower axis. However, this is not correct, the shower front is
not flat, it has a time distribution with a certain thickness and
curvature. The time distribution of particles in the shower front of an
air shower is the result of the different paths the particles take, and
the distance to the first interaction. Particles away from the shower
axis fall behind because they travel further. The thickness of the
shower front is also because ..


\section{\hisparc}

The \hisparc project aims to contribute to cosmic ray research by
creating a large network of detectors. Currently the largest distance
between two stations is \SI{1000}{\kilo\meter}.

Sensitive to high and low energies...

\cite{fokkema2012}


Size of network
Acceptance -> $10^{14} eV < E_{rec} < \inf$

Detector calibration
Energy calibration
Composition

Outline thesis

    Different types of analyses
    Time structure of showers
    Particles kinds and counts
    Various results..

Experiment contributions..
Open questions..

(then on to our contributions..)
