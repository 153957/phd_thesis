\chapter{Outreach}
\label{ch:outreach}

\section{Infopakket}

In late 2013 a new effort was started to provide Dutch high-school
physics teachers with easy to use course materials including assignments
and practical assignments.

The first edition contained xx documents, for various categories.


\section{Data access}

All data for the \hisparc stations is freely available to everyone.
Using \sapphire the raw data can be acquired. However, this requires some proficiency with \python. The Public Database is meant to ease data access.


\subsection{Download data form}

With the addition of the ESD to the Public Database it has become easier
to provide a web based download form to allow downloading data as csv.
This lowers the threshold for data access from requiring \python skills
to just opening a csv file.


\subsection{Coincidences}

To improve insight in the contributions of individual stations to the
\hisparc network we have added views that show the number of
coincidences between \hisparc stations in a day. If more stations are
online, more coincidence will occur. Unfortunately this partly due to
accidental coincidences, two unrelated events. However, when stations
are relatively close together the likelihood increases that the events
in the coincidences are from the same air shower.




\section{\jsparc}

Javascript library.


\subsection{Data retrieval tool}

Data analysis in any modern browser (IE9+, Firefox, Chrome, Safari)
Interpolation between different datasets. Can even make histograms.

What is it used for.
