\chapter{Single \hisparc station}
\label{ch:station}

\section{Station performance}

\subsection{Signal digitization}

The signal is sampled by two \adcs. To get good data and triggering the
\adcs need to be calibrated to have the same baseline and response
(gain). The \adcs normally have a range of \SIrange{-1}{1}{\volt}
translated into a 12-bit value. The \pmt outputs negative pulses, the
positive pulses are less important, but can be used to detect
reflections. An electric circuit in the \hisparc electronics modifies
the incoming signal to make the effective \adc range
\SIrange{-2}{0.1}{\volt}. The baseline of the signals is calibrated to
make an input of \SI{0}{\volt} correspond to \SI{200}{\adc} (30 in
HiSPARC iii). The gain is such to get a conversion factor of
\SI{-0.57}{\milli\volt\per\adc}. In a 2-detector station the \hisparc
electronics triggers when both detectors cross the low threshold of
\SI{30}{\milli\volt} within the trigger window. 30 mV / .57 = 53 above
baseline. So the low threshold is set to a value of \SI{253}{\adc}.
Either of the two \adcs can be the one to cross this threshold, so
proper \adc alignments is necessary to ensure they treat the signals
equally.

\adcs are stable and accurate when they have a constant temperature.
When \hisparc electronics are first connected the temperature of the
\adcs will be below operating temperature. When calibration is performed
immediately after powering the electronics it is likely to drift from
proper calibration as it continues to operate. Calibration may need to
be performed again.. (how much drift?).


\subsection{\adc sampling}

The \pmt signals are sampled every \SI{2.5}{\ns} by one of the
two \adcs.


\subsection{\mip/MPV}

The number of particles in a detector are determined by the pulse
integral, the integrated signal of the \pmt recorded by the \adc. When
making a histogram of all pulsintegrals this will not show discrete
values for the number of particles in each detection but a continuous
distribution. There is overlap in the pulse integrals from one particle
and the pulse integrals from two particles. The pulse integral histogram
does peak at a certain value, this is the most probable value for the
pulse for one particle, since it dominates the histogram. The value at
which this peak occurs is called the Most Probable Value (MPV).

A linear anti-correlation of several \SI{1}{\adc\per\kelvin} was found
between the MPV and the temperature of the \pmt (Bartels, Loran). Also
noted in the specifications as $\pm$
\SI{.5}{\percent\per\degreeCelsius}. The temperature of the \pmt can be
determined from the measured temperature outside and the solar
radiation. Best to determine the MPV in a period of constant
temperature, but with enough events for a good fit.
