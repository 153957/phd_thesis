\section{Timing}

\subsection{Fluorescence}

rise time - 0.9 ns, decay time - 2.1 ns, pulse width 2.5 ns.

\todo{deexcitation fast (2.1 ns), slow phosphorescence
      (14.2 ns), ratio?}


\subsection{Light transport and \pmt efficiency}

Similar to the model that predicts the signal transport efficiency in
the scintillator a models that calculates the transport time of these
photons has been made. Quantum efficiency of the \pmt causes not every
photon reaching the \pmt to produce a photoelectron. Multiple photons
are required for the \pmt to produce a significant signal that the
\hisparc electronics triggers. The model tracks the time at which the
15th photon producing a photoelectron reaches the \pmt. 15
photoelectrons are required for a signal of \SI{70}{\milli\volt} (high
threshold).

In the simulations for each particle within the bounds of the
scintillator a number is taken from this distribution simulate the
transport time of each particle.

\begin{figure}
    \centering
    % \usepackage{tikz}
% \usetikzlibrary{arrows}
% \usepackage{pgfplots}
% \pgfplotsset{compat=1.3}
% \usepackage[detect-family]{siunitx}
% \usepackage[eulergreek]{sansmath}
% \sisetup{text-sf=\sansmath}
% \usepackage{relsize}
%
\pgfkeysifdefined{/artist/width}
    {\pgfkeysgetvalue{/artist/width}{\defaultwidth}}
    {\def\defaultwidth{ .67\linewidth }}
%
%
\begin{sansmath}
\begin{tikzpicture}[
        font=\sffamily,
        every pin/.style={inner sep=2pt, font={\sffamily\smaller}},
        every label/.style={inner sep=2pt, font={\sffamily\smaller}},
        every pin edge/.style={<-, >=stealth', shorten <=2pt},
        pin distance=2.5ex,
    ]
    \begin{axis}[
            width=\defaultwidth,
            %
            title={  },
            %
            xlabel={ Transport time [\si{\nano\second}] },
            ylabel={ Counts },
            %
            xmin={ 1 },
            xmax={ 8 },
            ymin={ 0 },
            ymax={  },
            %
            xtick={  },
            ytick={  },
            %
            tick align=outside,
            max space between ticks=40,
            every tick/.style={},
        ]

        

        

        
            
            % Draw series plot
            \addplot[no markers,gray,const plot] coordinates {
                (1.0, 0.0)
                (1.1, 0.0)
                (1.2, 1.0)
                (1.3, 3.0)
                (1.4, 2.0)
                (1.5, 4.0)
                (1.6, 3.0)
                (1.7, 2.0)
                (1.8, 5.0)
                (1.9, 19.0)
                (2.0, 22.0)
                (2.1, 26.0)
                (2.2, 34.0)
                (2.3, 58.0)
                (2.4, 78.0)
                (2.5, 123.0)
                (2.6, 201.0)
                (2.7, 296.0)
                (2.8, 414.0)
                (2.9, 445.0)
                (3.0, 435.0)
                (3.1, 460.0)
                (3.2, 552.0)
                (3.3, 579.0)
                (3.4, 304.0)
                (3.5, 196.0)
                (3.6, 198.0)
                (3.7, 185.0)
                (3.8, 191.0)
                (3.9, 193.0)
                (4.0, 165.0)
                (4.1, 203.0)
                (4.2, 177.0)
                (4.3, 218.0)
                (4.4, 239.0)
                (4.5, 199.0)
                (4.6, 205.0)
                (4.7, 192.0)
                (4.8, 218.0)
                (4.9, 165.0)
                (5.0, 230.0)
                (5.1, 183.0)
                (5.2, 216.0)
                (5.3, 177.0)
                (5.4, 226.0)
                (5.5, 208.0)
                (5.6, 191.0)
                (5.7, 188.0)
                (5.8, 153.0)
                (5.9, 178.0)
                (6.0, 165.0)
                (6.1, 197.0)
                (6.2, 149.0)
                (6.3, 130.0)
                (6.4, 117.0)
                (6.5, 89.0)
                (6.6, 75.0)
                (6.7, 55.0)
                (6.8, 35.0)
                (6.9, 20.0)
                (7.0, 5.0)
                (7.1, 2.0)
                (7.2, 1.0)
                (7.3, 0.0)
                (7.4, 0.0)
                (7.5, 0.0)
                (7.6, 0.0)
                (7.7, 0.0)
                (7.8, 0.0)
                (7.9, 0.0)
                (8.0, 0.0)
            };
        
            
            % Draw series plot
            \addplot[no markers,black] coordinates {
                (2.5507, 0)
                (2.5507, 380.96)
                (3.4953, 380.96)
                (3.4953, 186.68)
                (6.463, 186.68)
                (6.463, 0)
            };
        

        

        

        

        

    \end{axis}
\end{tikzpicture}
\end{sansmath}
    \caption{\captitle{Signal transport time.} Time transport
             distribution in the detector.}
    \label{fig:transport_time}
\end{figure}


\subsection{\pmt transport, cable length}

The time it takes for the electron cascade in the \pmt to go from the
front window to the anode is dependent on .. the high voltage of pmt,
position electron hits, etc? Possible second pulse from ionization of
atoms in tube.

The signal from the \pmt is transported over a \SI{30}{\meter} long
coaxial cable to the \hisparc electronics before it is digitized. To
keep the detectors in sync the cables are precisely cut to the
same length. The signal in the cables propagates at 2/3 the speed of
light or \SI{.2}{\meter\per\nano\second}. Every \SI{.2}{\meter}
difference in cable length increases the offset by a nanosecond. Kinks
and twists in the cable can also increase the propagation time.

The offset between detectors can change over time due to aging of the
\pmt, changes in the high voltage and modifications to the cables.

Offsets are determined by looking at the arrival time differences
between two detectors over many events. All events in which a signal was
detected in the reference detector (usually detector 2, because of its
central location in 4-detector stations) and in another detector are
used. Multiple events are needed to even out time differences due to the
arrival direction of the showers, transport time and the shower front.
The time difference distribution between two detectors includes a
background of accidental coincidences.

To determine the distribution of the mean offsets the distributions for
a day of data for all stations have been fit by Gaussians, the found
averages are then the distribution of the offsets. These have also been
fit by a Gaussian \cref{fig:detector_offset_distribution}. In
simulations a detector offset is drawn from this distribution and then
fixed for that detector for the entire simulation.

\todo{relative to detector 2.. different sigma if offset also given to
      detector 2?}

\begin{figure}
    \centering
    % \usepackage{tikz}
% \usetikzlibrary{arrows,external}
% \usepackage{pgfplots}
% \pgfplotsset{compat=1.3}
% \usepackage[detect-family]{siunitx}
% \usepackage[eulergreek]{sansmath}
% \sisetup{text-sf=\sansmath}
% \usepackage{relsize}
%
    \tikzsetnextfilename{ externalized-detector_offset_distribution_20140102 }
\pgfkeysifdefined{/artist/width}
    {\pgfkeysgetvalue{/artist/width}{\defaultwidth}}
    {\def\defaultwidth{ .67\linewidth }}
%
%
\begin{sansmath}
\begin{tikzpicture}[
        font=\sffamily,
        every pin/.style={inner sep=2pt, font={\sffamily\smaller}},
        every label/.style={inner sep=2pt, font={\sffamily\smaller}},
        every pin edge/.style={<-, >=stealth', shorten <=2pt},
        pin distance=2.5ex,
    ]
    \begin{axis}[
            axis background/.style={  },
            xmode=normal,
            ymode=normal,
            width=\defaultwidth,
            scale only axis,
            axis equal=false,
            %
            title={  },
            %
            xlabel={ $\Delta t$ [ns] },
            ylabel={ Occurrence },
            %
            xmin={  },
            xmax={  },
            ymin={ 0 },
            ymax={  },
            %
            xtick={  },
            ytick={  },
            %
            tick align=outside,
            max space between ticks=40,
            every tick/.style={},
            axis on top,
            point meta min={  },
            point meta max={  },
                colormap={coolwarm}{
                    rgb255(0cm)=( 59, 76,192);
                    rgb255(1cm)=( 98,130,234);
                    rgb255(2cm)=(141,176,254);
                    rgb255(3cm)=(184,208,249);
                    rgb255(4cm)=(221,221,221);
                    rgb255(5cm)=(245,196,173);
                    rgb255(6cm)=(244,154,123);
                    rgb255(7cm)=(222, 96, 77);
                    rgb255(8cm)=(180,  4, 38)},
        ]

        


    
    % Draw series plot
    \addplot[no markers,solid,const plot] coordinates {
            (-49.375, 0)
            (-48.125, 1)
            (-46.875, 0)
            (-45.625, 0)
            (-44.375, 0)
            (-43.125, 0)
            (-41.875, 0)
            (-40.625, 1)
            (-39.375, 1)
            (-38.125, 0)
            (-36.875, 0)
            (-35.625, 0)
            (-34.375, 0)
            (-33.125, 0)
            (-31.875, 0)
            (-30.625, 0)
            (-29.375, 0)
            (-28.125, 0)
            (-26.875, 0)
            (-25.625, 0)
            (-24.375, 0)
            (-23.125, 0)
            (-21.875, 0)
            (-20.625, 0)
            (-19.375, 0)
            (-18.125, 0)
            (-16.875, 0)
            (-15.625, 0)
            (-14.375, 0)
            (-13.125, 0)
            (-11.875, 0)
            (-10.625, 0)
            (-9.375, 1)
            (-8.125, 0)
            (-6.875, 1)
            (-5.625, 3)
            (-4.375, 7)
            (-3.125, 9)
            (-1.875, 11)
            (-0.625, 15)
            (0.625, 13)
            (1.875, 11)
            (3.125, 4)
            (4.375, 3)
            (5.625, 2)
            (6.875, 0)
            (8.125, 0)
            (9.375, 0)
            (10.625, 0)
            (11.875, 0)
            (13.125, 0)
            (14.375, 1)
            (15.625, 1)
            (16.875, 0)
            (18.125, 0)
            (19.375, 0)
            (20.625, 0)
            (21.875, 0)
            (23.125, 0)
            (24.375, 0)
            (25.625, 0)
            (26.875, 0)
            (28.125, 0)
            (29.375, 0)
            (30.625, 1)
            (31.875, 0)
            (33.125, 0)
            (34.375, 0)
            (35.625, 0)
            (36.875, 0)
            (38.125, 0)
            (39.375, 0)
            (40.625, 0)
            (41.875, 0)
            (43.125, 0)
            (44.375, 0)
            (45.625, 0)
            (46.875, 0)
            (48.125, 0)
    };

    
    % Draw series plot
    \addplot[no markers,gray] coordinates {
            (-48.75, 1.09919865298e-64)
            (-48.65, 2.03061331221e-64)
            (-48.55, 3.74654989725e-64)
            (-48.45, 6.90381370184e-64)
            (-48.35, 1.27057368552e-63)
            (-48.25, 2.33541411175e-63)
            (-48.15, 4.28727354586e-63)
            (-48.05, 7.86052835737e-63)
            (-47.95, 1.43938023192e-62)
            (-47.85, 2.63240422725e-62)
            (-47.75, 4.80820389252e-62)
            (-47.65, 8.77134962692e-62)
            (-47.55, 1.59809730667e-61)
            (-47.45, 2.90799240084e-61)
            (-47.35, 5.28489746223e-61)
            (-47.25, 9.59252885553e-61)
            (-47.15, 1.73893325758e-60)
            (-47.05, 3.1483714346e-60)
            (-46.95, 5.6930141555e-60)
            (-46.85, 1.0281389553e-59)
            (-46.75, 1.85444781188e-59)
            (-46.65, 3.34064762129e-59)
            (-46.55, 6.01035286714e-59)
            (-46.45, 1.07999693518e-58)
            (-46.35, 1.93819882736e-58)
            (-46.25, 3.47398053262e-58)
            (-46.15, 6.21884431865e-58)
            (-46.05, 1.11184754098e-57)
            (-45.95, 1.98533612669e-57)
            (-45.85, 3.5405937749e-57)
            (-45.75, 6.30625325864e-57)
            (-45.65, 1.12181169966e-56)
            (-45.55, 1.99306642285e-56)
            (-45.45, 3.53652580026e-56)
            (-45.35, 6.26736718542e-56)
            (-45.25, 1.10929408834e-55)
            (-45.15, 1.96092736961e-55)
            (-45.05, 3.46201996776e-55)
            (-44.95, 6.10451097853e-55)
            (-44.85, 1.0750420214e-54)
            (-44.75, 1.89083334317e-54)
            (-44.65, 3.32150044044e-54)
            (-44.55, 5.82731677528e-54)
            (-44.45, 1.02107158479e-53)
            (-44.35, 1.78688665041e-53)
            (-44.25, 3.12313723848e-53)
            (-44.15, 5.45178077099e-53)
            (-44.05, 9.50471171426e-53)
            (-43.95, 1.65498005712e-52)
            (-43.85, 2.87805991802e-52)
            (-43.75, 4.9987353941e-52)
            (-43.65, 8.67109043343e-52)
            (-43.55, 1.50224419203e-51)
            (-43.45, 2.59932510966e-51)
            (-43.35, 4.49193973631e-51)
            (-43.25, 7.75283409878e-51)
            (-43.15, 1.33641182808e-50)
            (-43.05, 2.30077092932e-50)
            (-42.95, 3.95603114863e-50)
            (-42.85, 6.79358907479e-50)
            (-42.75, 1.1651775176e-49)
            (-42.65, 1.99589719987e-49)
            (-42.55, 3.41458157108e-49)
            (-42.45, 5.8343175845e-49)
            (-42.35, 9.95625221998e-49)
            (-42.25, 1.6968949794e-48)
            (-42.15, 2.88846620044e-48)
            (-42.05, 4.91058084896e-48)
            (-41.95, 8.33780419744e-48)
            (-41.85, 1.41391652046e-47)
            (-41.75, 2.39468887291e-47)
            (-41.65, 4.05067756015e-47)
            (-41.55, 6.84320424733e-47)
            (-41.45, 1.15463463683e-46)
            (-41.35, 1.94573150408e-46)
            (-41.25, 3.27472240159e-46)
            (-41.15, 5.5045183163e-46)
            (-41.05, 9.24096663398e-46)
            (-40.95, 1.54941848763e-45)
            (-40.85, 2.59461733695e-45)
            (-40.75, 4.33941462365e-45)
            (-40.65, 7.24840132803e-45)
            (-40.55, 1.20922346155e-44)
            (-40.45, 2.0147638455e-44)
            (-40.35, 3.35270223187e-44)
            (-40.25, 5.57210213977e-44)
            (-40.15, 9.24903460485e-44)
            (-40.05, 1.53329948751e-43)
            (-39.95, 2.53869675101e-43)
            (-39.85, 4.19805297313e-43)
            (-39.75, 6.93327221982e-43)
            (-39.65, 1.14362025958e-42)
            (-39.55, 1.88399037267e-42)
            (-39.45, 3.09976498727e-42)
            (-39.35, 5.09368528338e-42)
            (-39.25, 8.3596616215e-42)
            (-39.15, 1.37024598901e-41)
            (-39.05, 2.24316712174e-41)
            (-38.95, 3.66756629363e-41)
            (-38.85, 5.9889069455e-41)
            (-38.75, 9.76720736217e-41)
            (-38.65, 1.59091325966e-40)
            (-38.55, 2.58806896849e-40)
            (-38.45, 4.20492680055e-40)
            (-38.35, 6.82329715123e-40)
            (-38.25, 1.10581730137e-39)
            (-38.15, 1.78988746498e-39)
            (-38.05, 2.89348558532e-39)
            (-37.95, 4.67164862588e-39)
            (-37.85, 7.53307459266e-39)
            (-37.75, 1.21318663134e-38)
            (-37.65, 1.95135468132e-38)
            (-37.55, 3.13471501251e-38)
            (-37.45, 5.02936509637e-38)
            (-37.35, 8.05900664159e-38)
            (-37.25, 1.28974280686e-37)
            (-37.15, 2.06147449343e-37)
            (-37.05, 3.29083469524e-37)
            (-36.95, 5.24671434297e-37)
            (-36.85, 8.35452999343e-37)
            (-36.75, 1.32864780944e-36)
            (-36.65, 2.11033294887e-36)
            (-36.55, 3.34769076342e-36)
            (-36.45, 5.30387083212e-36)
            (-36.35, 8.39254723706e-36)
            (-36.25, 1.32631872258e-35)
            (-36.15, 2.09341463846e-35)
            (-36.05, 3.30001461797e-35)
            (-35.95, 5.19552837634e-35)
            (-35.85, 8.16952543318e-35)
            (-35.75, 1.2829720277e-34)
            (-35.65, 2.01229105513e-34)
            (-35.55, 3.15222824597e-34)
            (-35.45, 4.93171267488e-34)
            (-35.35, 7.70603765717e-34)
            (-35.25, 1.20259041501e-33)
            (-35.15, 1.87437981919e-33)
            (-35.05, 2.91776769719e-33)
            (-34.95, 4.53625097499e-33)
            (-34.85, 7.04363248518e-33)
            (-34.75, 1.09231915984e-32)
            (-34.65, 1.69182588622e-32)
            (-34.55, 2.61706814033e-32)
            (-34.45, 4.0432225407e-32)
            (-34.35, 6.23869151458e-32)
            (-34.25, 9.61418838967e-32)
            (-34.15, 1.47973857231e-31)
            (-34.05, 2.27462921078e-31)
            (-33.95, 3.49212258069e-31)
            (-33.85, 5.35453298592e-31)
            (-33.75, 8.19987000133e-31)
            (-33.65, 1.25413874978e-30)
            (-33.55, 1.91574392319e-30)
            (-33.45, 2.92268881206e-30)
            (-33.35, 4.45328969877e-30)
            (-33.25, 6.77692328778e-30)
            (-33.15, 1.03000051452e-29)
            (-33.05, 1.56349164286e-29)
            (-32.95, 2.37031976155e-29)
            (-32.85, 3.58898431946e-29)
            (-32.75, 5.42737006556e-29)
            (-32.65, 8.19710615971e-29)
            (-32.55, 1.2364738466e-28)
            (-32.45, 1.86278426356e-28)
            (-32.35, 2.80280856929e-28)
            (-32.25, 4.21189526569e-28)
            (-32.15, 6.32142428562e-28)
            (-32.05, 9.47557488838e-28)
            (-31.95, 1.4185656736e-27)
            (-31.85, 2.12102884807e-27)
            (-31.75, 3.16735654964e-27)
            (-31.65, 4.72389879313e-27)
            (-31.55, 7.03651252221e-27)
            (-31.45, 1.04680935647e-26)
            (-31.35, 1.55536018389e-26)
            (-31.25, 2.3080627346e-26)
            (-31.15, 3.42071972132e-26)
            (-31.05, 5.06338119595e-26)
            (-30.95, 7.48543438606e-26)
            (-30.85, 1.10521467256e-25)
            (-30.75, 1.62978203533e-25)
            (-30.65, 2.40030075318e-25)
            (-30.55, 3.53065308671e-25)
            (-30.45, 5.18677827378e-25)
            (-30.35, 7.61015617825e-25)
            (-30.25, 1.11517418309e-24)
            (-30.15, 1.63209372238e-24)
            (-30.05, 2.38561706349e-24)
            (-29.95, 3.48264831838e-24)
            (-29.85, 5.07775520869e-24)
            (-29.75, 7.39413116067e-24)
            (-29.65, 1.07536473482e-23)
            (-29.55, 1.56198795655e-23)
            (-29.45, 2.26596304224e-23)
            (-29.35, 3.28307809933e-23)
            (-29.25, 4.7507574311e-23)
            (-29.15, 6.86590435412e-23)
            (-29.05, 9.91027925873e-23)
            (-28.95, 1.4286547398e-22)
            (-28.85, 2.05694145425e-22)
            (-28.75, 2.95780695394e-22)
            (-28.65, 4.2478675915e-22)
            (-28.55, 6.09291847268e-22)
            (-28.45, 8.72836729767e-22)
            (-28.35, 1.24880293413e-21)
            (-28.25, 1.78446509741e-21)
            (-28.15, 2.54668632327e-21)
            (-28.05, 3.62991206647e-21)
            (-27.95, 5.16737528349e-21)
            (-27.85, 7.34678199777e-21)
            (-27.75, 1.04322396045e-20)
            (-27.65, 1.47948717176e-20)
            (-27.55, 2.09555037348e-20)
            (-27.45, 2.96440991147e-20)
            (-27.35, 4.18824095091e-20)
            (-27.25, 5.90987522222e-20)
            (-27.15, 8.32871909467e-20)
            (-27.05, 1.17228003463e-19)
            (-26.95, 1.64792625385e-19)
            (-26.85, 2.31364875506e-19)
            (-26.75, 3.24422026515e-19)
            (-26.65, 4.54335307935e-19)
            (-26.55, 6.35471232409e-19)
            (-26.45, 8.87704766486e-19)
            (-26.35, 1.23849557483e-18)
            (-26.25, 1.72573305237e-18)
            (-26.15, 2.40162958834e-18)
            (-26.05, 3.33804113648e-18)
            (-25.95, 4.63372860575e-18)
            (-25.85, 6.42425474326e-18)
            (-25.75, 8.8954549449e-18)
            (-25.65, 1.2301747973e-17)
            (-25.55, 1.69909920577e-17)
            (-25.45, 2.343818074e-17)
            (-25.35, 3.22910607373e-17)
            (-25.25, 4.44318068611e-17)
            (-25.15, 6.10602935985e-17)
            (-25.05, 8.38063743142e-17)
            (-24.95, 1.14881068692e-16)
            (-24.85, 1.57279866606e-16)
            (-24.75, 2.15055744529e-16)
            (-24.65, 2.93685303363e-16)
            (-24.55, 4.00559125994e-16)
            (-24.45, 5.45637610078e-16)
            (-24.35, 7.42326929483e-16)
            (-24.25, 1.00864742486e-15)
            (-24.15, 1.36878993437e-15)
            (-24.05, 1.85518605953e-15)
            (-23.95, 2.51125833994e-15)
            (-23.85, 3.39506866615e-15)
            (-23.75, 4.58415172451e-15)
            (-23.65, 6.18190966785e-15)
            (-23.55, 8.32606077506e-15)
            (-23.45, 1.1199786635e-14)
            (-23.35, 1.50464197114e-14)
            (-23.25, 2.01887693101e-14)
            (-23.15, 2.70545160402e-14)
            (-23.05, 3.62095346987e-14)
            (-22.95, 4.84015610884e-14)
            (-22.85, 6.46173353184e-14)
            (-22.75, 8.61572782139e-14)
            (-22.65, 1.14732946592e-13)
            (-22.55, 1.52594041107e-13)
            (-22.45, 2.02693700788e-13)
            (-22.35, 2.68903329926e-13)
            (-22.25, 3.56291415823e-13)
            (-22.15, 4.71484899771e-13)
            (-22.05, 6.23136897788e-13)
            (-21.95, 8.22531233457e-13)
            (-21.85, 1.08436272152e-12)
            (-21.75, 1.42774290335e-12)
            (-21.65, 1.87749458008e-12)
            (-21.55, 2.4658157616e-12)
            (-21.45, 3.23441548183e-12)
            (-21.35, 4.23725148416e-12)
            (-21.25, 5.54403450221e-12)
            (-21.15, 7.2447075489e-12)
            (-21.05, 9.45516351192e-12)
            (-20.95, 1.23245332013e-11)
            (-20.85, 1.60444622245e-11)
            (-20.75, 2.08609028975e-11)
            (-20.65, 2.70890820445e-11)
            (-20.55, 3.51324733955e-11)
            (-20.45, 4.55068122321e-11)
            (-20.35, 5.88704496096e-11)
            (-20.25, 7.60626657094e-11)
            (-20.15, 9.81519610973e-11)
            (-20.05, 1.26496838496e-10)
            (-19.95, 1.62822197592e-10)
            (-19.85, 2.09315217471e-10)
            (-19.75, 2.68745527191e-10)
            (-19.65, 3.44615603146e-10)
            (-19.55, 4.41348728556e-10)
            (-19.45, 5.64523561842e-10)
            (-19.35, 7.21166454073e-10)
            (-19.25, 9.20115212496e-10)
            (-19.15, 1.17247112476e-09)
            (-19.05, 1.4921598536e-09)
            (-18.95, 1.89662642441e-09)
            (-18.85, 2.4076951246e-09)
            (-18.75, 3.05263191358e-09)
            (-18.65, 3.86545514104e-09)
            (-18.55, 4.88855027037e-09)
            (-18.45, 6.17465623581e-09)
            (-18.35, 7.78930541935e-09)
            (-18.25, 9.81381648058e-09)
            (-18.15, 1.23489599564e-08)
            (-18.05, 1.55194413095e-08)
            (-17.95, 1.94793756979e-08)
            (-17.85, 2.44189640471e-08)
            (-17.75, 3.05726220553e-08)
            (-17.65, 3.82288637605e-08)
            (-17.55, 4.77423006331e-08)
            (-17.45, 5.95481874539e-08)
            (-17.35, 7.4180029376e-08)
            (-17.25, 9.22908627037e-08)
            (-17.15, 1.14678937572e-07)
            (-17.05, 1.4231866673e-07)
            (-16.95, 1.76397864325e-07)
            (-16.85, 2.18362485756e-07)
            (-16.75, 2.69970298528e-07)
            (-16.65, 3.33355169674e-07)
            (-16.55, 4.11103953065e-07)
            (-16.45, 5.06348306358e-07)
            (-16.35, 6.22874169367e-07)
            (-16.25, 7.65252101464e-07)
            (-16.15, 9.38992214226e-07)
            (-16.05, 1.15072805701e-06)
            (-15.95, 1.40843452887e-06)
            (-15.85, 1.72168571279e-06)
            (-15.75, 2.10195947144e-06)
            (-15.65, 2.56299672347e-06)
            (-15.55, 3.12122455203e-06)
            (-15.45, 3.79625370258e-06)
            (-15.35, 4.61146262425e-06)
            (-15.25, 5.59468202196e-06)
            (-15.15, 6.77899593811e-06)
            (-15.05, 8.20367769987e-06)
            (-14.95, 9.91528167905e-06)
            (-14.85, 1.19689147463e-05)
            (-14.75, 1.44297145925e-05)
            (-14.65, 1.73745657697e-05)
            (-14.55, 2.08940884128e-05)
            (-14.45, 2.50949391703e-05)
            (-14.35, 3.01024689454e-05)
            (-14.25, 3.60637876601e-05)
            (-14.15, 4.31512924511e-05)
            (-14.05, 5.15667225187e-05)
            (-13.95, 6.15458113326e-05)
            (-13.85, 7.33636150712e-05)
            (-13.75, 8.73406050929e-05)
            (-13.65, 0.000103849621917)
            (-13.55, 0.000123323798689)
            (-13.45, 0.000146265573404)
            (-13.35, 0.000173256921334)
            (-13.25, 0.000204970952105)
            (-13.15, 0.000242185029722)
            (-13.05, 0.000285795588577)
            (-12.95, 0.000336834833941)
            (-12.85, 0.000396489531792)
            (-12.75, 0.000466122109788)
            (-12.65, 0.000547294308875)
            (-12.55, 0.00064179364316)
            (-12.45, 0.00075166294429)
            (-12.35, 0.000879233285357)
            (-12.25, 0.00102716059828)
            (-12.15, 0.00119846631724)
            (-12.05, 0.00139658239922)
            (-11.95, 0.00162540108996)
            (-11.85, 0.00188932982052)
            (-11.75, 0.00219335163447)
            (-11.65, 0.00254309155912)
            (-11.55, 0.0029448893449)
            (-11.45, 0.0034058790054)
            (-11.35, 0.00393407559498)
            (-11.25, 0.00453846966177)
            (-11.15, 0.0052291298097)
            (-11.05, 0.00601731379441)
            (-10.95, 0.00691558856204)
            (-10.85, 0.00793795961849)
            (-10.75, 0.00910001008673)
            (-10.65, 0.0104190497721)
            (-10.55, 0.0119142745086)
            (-10.45, 0.0136069360022)
            (-10.35, 0.0155205223197)
            (-10.25, 0.0176809490932)
            (-10.15, 0.0201167614192)
            (-10.05, 0.0228593463264)
            (-9.95, 0.025943155573)
            (-9.85, 0.0294059383989)
            (-9.75, 0.0332889837187)
            (-9.65, 0.0376373710796)
            (-9.55, 0.0425002295362)
            (-9.45, 0.0479310034096)
            (-9.35, 0.0539877236945)
            (-9.25, 0.0607332836685)
            (-9.15, 0.06823571703)
            (-9.05, 0.0765684766568)
            (-8.95, 0.0858107118309)
            (-8.85, 0.0960475415227)
            (-8.75, 0.107370321069)
            (-8.65, 0.119876899314)
            (-8.55, 0.133671863035)
            (-8.45, 0.148866765182)
            (-8.35, 0.165580333254)
            (-8.25, 0.183938653846)
            (-8.15, 0.204075329196)
            (-8.05, 0.226131601357)
            (-7.95, 0.250256439399)
            (-7.85, 0.27660658491)
            (-7.75, 0.305346550946)
            (-7.65, 0.336648569453)
            (-7.55, 0.370692482166)
            (-7.45, 0.407665569985)
            (-7.35, 0.44776231587)
            (-7.25, 0.491184096418)
            (-7.15, 0.538138797455)
            (-7.05, 0.588840349209)
            (-6.95, 0.643508176947)
            (-6.85, 0.702366563345)
            (-6.75, 0.765643919301)
            (-6.65, 0.833571960462)
            (-6.55, 0.906384787322)
            (-6.45, 0.984317867478)
            (-6.35, 1.06760691935)
            (-6.25, 1.15648669761)
            (-6.15, 1.25118968129)
            (-6.05, 1.35194466686)
            (-5.95, 1.4589752692)
            (-5.85, 1.57249833483)
            (-5.75, 1.69272227273)
            (-5.65, 1.81984530933)
            (-5.55, 1.95405367546)
            (-5.45, 2.09551973415)
            (-5.35, 2.24440005971)
            (-5.25, 2.40083347918)
            (-5.15, 2.56493908912)
            (-5.05, 2.73681426118)
            (-4.95, 2.91653265128)
            (-4.85, 3.10414222835)
            (-4.75, 3.29966333907)
            (-4.65, 3.50308682618)
            (-4.55, 3.71437221847)
            (-4.45, 3.93344601084)
            (-4.35, 4.1602000537)
            (-4.25, 4.39449007044)
            (-4.15, 4.63613432229)
            (-4.05, 4.88491243927)
            (-3.95, 5.14056443578)
            (-3.85, 5.40278992843)
            (-3.75, 5.67124757317)
            (-3.65, 5.94555473746)
            (-3.55, 6.22528742203)
            (-3.45, 6.50998044518)
            (-3.35, 6.79912790101)
            (-3.25, 7.09218390074)
            (-3.15, 7.38856360448)
            (-3.05, 7.68764454837)
            (-2.95, 7.98876826953)
            (-2.85, 8.29124222876)
            (-2.75, 8.5943420283)
            (-2.65, 8.89731391909)
            (-2.55, 9.19937758922)
            (-2.45, 9.49972922242)
            (-2.35, 9.79754481267)
            (-2.25, 10.0919837182)
            (-2.15, 10.3821924353)
            (-2.05, 10.6673085701)
            (-1.95, 10.946464984)
            (-1.85, 11.2187940853)
            (-1.75, 11.4834322392)
            (-1.65, 11.7395242649)
            (-1.55, 11.986227988)
            (-1.45, 12.2227188143)
            (-1.35, 12.4481942912)
            (-1.25, 12.6618786209)
            (-1.15, 12.8630270911)
            (-1.05, 13.050930387)
            (-0.949999999999, 13.2249187503)
            (-0.849999999999, 13.3843659507)
            (-0.749999999999, 13.5286930375)
            (-0.649999999999, 13.6573718396)
            (-0.549999999999, 13.7699281847)
            (-0.449999999999, 13.865944809)
            (-0.349999999999, 13.9450639343)
            (-0.249999999999, 14.0069894887)
            (-0.149999999999, 14.0514889525)
            (-0.0499999999993, 14.0783948125)
            (0.0500000000007, 14.0876056126)
            (0.150000000001, 14.0790865912)
            (0.250000000001, 14.0528698996)
            (0.350000000001, 14.0090544001)
            (0.450000000001, 13.9478050447)
            (0.550000000001, 13.8693518407)
            (0.650000000001, 13.7739884116)
            (0.750000000001, 13.6620701664)
            (0.850000000001, 13.5340120933)
            (0.950000000001, 13.3902861959)
            (1.05, 13.2314185964)
            (1.15, 13.0579863271)
            (1.25, 12.870613842)
            (1.35, 12.6699692737)
            (1.45, 12.4567604699)
            (1.55, 12.2317308415)
            (1.65, 11.9956550551)
            (1.75, 11.7493346075)
            (1.85, 11.4935933138)
            (1.95, 11.2292727483)
            (2.05, 10.9572276697)
            (2.15, 10.6783214675)
            (2.25, 10.3934216624)
            (2.35, 10.1033954921)
            (2.45, 9.80910561443)
            (2.55, 9.51140595572)
            (2.65, 9.2111377319)
            (2.75, 8.90912566642)
            (2.85, 8.60617442732)
            (2.95, 8.30306530297)
            (3.05, 8.00055313327)
            (3.15, 7.69936351043)
            (3.25, 7.40019026058)
            (3.35, 7.10369321459)
            (3.45, 6.81049627378)
            (3.55, 6.52118577332)
            (3.65, 6.2363091435)
            (3.75, 5.95637386654)
            (3.85, 5.68184672401)
            (3.95, 5.41315332772)
            (4.05, 5.15067792497)
            (4.15, 4.89476346673)
            (4.25, 4.645711926)
            (4.35, 4.40378485182)
            (4.45, 4.16920414317)
            (4.55, 3.94215302581)
            (4.65, 3.72277721447)
            (4.75, 3.51118624191)
            (4.85, 3.30745493593)
            (4.95, 3.11162502548)
            (5.05, 2.92370685658)
            (5.15, 2.74368119921)
            (5.25, 2.57150112653)
            (5.35, 2.4070939484)
            (5.45, 2.25036318154)
            (5.55, 2.10119053989)
            (5.65, 1.95943792917)
            (5.75, 1.82494943084)
            (5.85, 1.69755326177)
            (5.95, 1.57706369695)
            (6.05, 1.46328294372)
            (6.15, 1.35600295732)
            (6.25, 1.25500718874)
            (6.35, 1.16007225697)
            (6.45, 1.07096953907)
            (6.55, 0.987466672734)
            (6.65, 0.909328966896)
            (6.75, 0.836320717332)
            (6.85, 0.768206425082)
            (6.95, 0.704751916568)
            (7.05, 0.645725365179)
            (7.15, 0.59089821497)
            (7.25, 0.540046007861)
            (7.35, 0.492949116456)
            (7.45, 0.449393385192)
            (7.55, 0.409170683077)
            (7.65, 0.372079371739)
            (7.75, 0.337924692898)
            (7.85, 0.30651907966)
            (7.95, 0.277682396325)
            (8.05, 0.251242111506)
            (8.15, 0.227033409543)
            (8.25, 0.204899245185)
            (8.35, 0.184690346569)
            (8.45, 0.166265171436)
            (8.55, 0.149489821467)
            (8.65, 0.134237919472)
            (8.75, 0.120390454019)
            (8.85, 0.1078355959)
            (8.95, 0.0964684906114)
            (9.05, 0.0861910308163)
            (9.15, 0.0769116124863)
            (9.25, 0.0685448781942)
            (9.35, 0.0610114507516)
            (9.45, 0.0542376601309)
            (9.55, 0.0481552663476)
            (9.65, 0.0427011807196)
            (9.75, 0.0378171876676)
            (9.85, 0.0334496689745)
            (9.95, 0.0295493321851)
            (10.05, 0.0260709446028)
            (10.15, 0.0229730741251)
            (10.25, 0.0202178379593)
            (10.35, 0.0177706600746)
            (10.45, 0.0156000380697)
            (10.55, 0.0136773199801)
            (10.65, 0.0119764914008)
            (10.75, 0.0104739731723)
            (10.85, 0.00914842975766)
            (10.95, 0.00798058833598)
            (11.05, 0.00695306854593)
            (11.15, 0.00605022273346)
            (11.25, 0.00525798649)
            (11.35, 0.00456373921025)
            (11.45, 0.00395617435134)
            (11.55, 0.00342517903687)
            (11.65, 0.00296172261962)
            (11.75, 0.00255775379424)
            (11.85, 0.0022061058359)
            (11.95, 0.00190040953126)
            (12.05, 0.00163501336424)
            (12.15, 0.00140491051928)
            (12.25, 0.0012056722696)
            (12.35, 0.00103338732586)
            (12.45, 0.000884606731371)
            (12.55, 0.000756293903102)
            (12.65, 0.000645779433055)
            (12.75, 0.000550720280639)
            (12.85, 0.000469063004527)
            (12.95, 0.000399010700593)
            (13.05, 0.000338993331291)
            (13.15, 0.000287641150695)
            (13.25, 0.000243760948236)
            (13.35, 0.000206314852782)
            (13.45, 0.000174401456883)
            (13.55, 0.000147239038677)
            (13.65, 0.000124150675972)
            (13.75, 0.000104551063369)
            (13.85, 8.79348588068e-05)
            (13.95, 7.38664006674e-05)
            (14.05, 6.19706504742e-05)
            (14.15, 5.19252292464e-05)
            (14.25, 4.34534277744e-05)
            (14.35, 3.63180824177e-05)
            (14.45, 3.03162185519e-05)
            (14.55, 2.52743735042e-05)
            (14.65, 2.10445197665e-05)
            (14.75, 1.75005174763e-05)
            (14.85, 1.45350326649e-05)
            (14.95, 1.20568646099e-05)
            (15.05, 9.98863184931e-06)
            (15.15, 8.26477204739e-06)
            (15.25, 6.82981599402e-06)
            (15.35, 5.63690060945e-06)
            (15.45, 4.64648994806e-06)
            (15.55, 3.82527689314e-06)
            (15.65, 3.14524153903e-06)
            (15.75, 2.58284520543e-06)
            (15.85, 2.11834165177e-06)
            (15.95, 1.73518938729e-06)
            (16.05, 1.41955103447e-06)
            (16.15, 1.1598675248e-06)
            (16.25, 9.464965113e-07)
            (16.35, 7.71405794544e-07)
            (16.45, 6.27913798173e-07)
            (16.55, 5.10470215084e-07)
            (16.65, 4.14470893939e-07)
            (16.75, 3.36101862537e-07)
            (16.85, 2.72208104283e-07)
            (16.95, 2.20183328905e-07)
            (17.05, 1.77877520229e-07)
            (17.15, 1.43519512268e-07)
            (17.25, 1.15652249318e-07)
            (17.35, 9.30787341303e-08)
            (17.45, 7.48169678394e-08)
            (17.55, 6.00624424353e-08)
            (17.65, 4.81569668054e-08)
            (17.75, 3.85627956836e-08)
            (17.85, 3.08411915346e-08)
            (17.95, 2.46346862861e-08)
            (18.05, 1.96524262055e-08)
            (18.15, 1.56580819901e-08)
            (18.25, 1.24598898203e-08)
            (18.35, 9.90245988304e-09)
            (18.45, 7.86004860899e-09)
            (18.55, 6.2310411962e-09)
            (18.65, 4.93343333293e-09)
            (18.75, 3.90113658191e-09)
            (18.85, 3.08096168792e-09)
            (18.95, 2.43015900662e-09)
            (19.05, 1.9144160823e-09)
            (19.15, 1.5062297728e-09)
            (19.25, 1.18358478091e-09)
            (19.35, 9.28882470316e-10)
            (19.45, 7.28073815667e-10)
            (19.55, 5.69958596478e-10)
            (19.65, 4.45619774761e-10)
            (19.75, 3.47967635246e-10)
            (19.85, 2.71372914682e-10)
            (19.95, 2.11371971039e-10)
            (20.05, 1.64430185116e-10)
            (20.15, 1.27752363839e-10)
            (20.25, 9.91310244451e-11)
            (20.35, 7.68251637097e-11)
            (20.45, 5.94635242266e-11)
            (20.55, 4.59675170775e-11)
            (20.65, 3.5489893633e-11)
            (20.75, 2.73660174702e-11)
            (20.85, 2.1075202322e-11)
            (20.95, 1.62100799087e-11)
            (21.05, 1.24523640636e-11)
            (21.15, 9.55370250764e-12)
            (21.25, 7.3205695057e-12)
            (21.35, 5.60236336876e-12)
            (21.45, 4.2820420668e-12)
            (21.55, 3.26876602855e-12)
            (21.65, 2.49212597985e-12)
            (21.75, 1.89762065985e-12)
            (21.85, 1.44311871168e-12)
            (21.95, 1.09609441845e-12)
            (22.05, 8.31471035655e-13)
            (22.15, 6.29940505449e-13)
            (22.25, 4.76656143076e-13)
            (22.35, 3.60216917076e-13)
            (22.45, 2.71879376931e-13)
            (22.55, 2.04947052832e-13)
            (22.65, 1.54298015171e-13)
            (22.75, 1.16019837287e-13)
            (22.85, 8.71279338124e-14)
            (22.95, 6.53485309075e-14)
            (23.05, 4.89516677839e-14)
            (23.15, 3.66228727043e-14)
            (23.25, 2.73646925181e-14)
            (23.35, 2.04212343036e-14)
            (23.45, 1.52204206198e-14)
            (23.55, 1.13298607225e-14)
            (23.65, 8.42317334483e-15)
            (23.75, 6.25432083675e-15)
            (23.85, 4.63807561131e-15)
            (23.95, 3.43517402458e-15)
            (24.05, 2.54104825488e-15)
            (24.15, 1.87728552834e-15)
            (24.25, 1.38516338369e-15)
            (24.35, 1.02076300905e-15)
            (24.45, 7.51280462418e-16)
            (24.55, 5.52245920977e-16)
            (24.65, 4.05430287327e-16)
            (24.75, 2.97271388819e-16)
            (24.85, 2.17692409123e-16)
            (24.95, 1.5921600024e-16)
            (25.05, 1.16300982614e-16)
            (25.15, 8.48463783087e-17)
            (25.25, 6.18210653081e-17)
            (25.35, 4.49876088711e-17)
            (25.45, 3.26965994679e-17)
            (25.55, 2.37337035997e-17)
            (25.65, 1.72060702589e-17)
            (25.75, 1.24580801351e-17)
            (25.85, 9.00894216249e-18)
            (25.95, 6.50653436537e-18)
            (26.05, 4.69330639788e-18)
            (26.15, 3.38112586367e-18)
            (26.25, 2.4327474084e-18)
            (26.35, 1.74817920474e-18)
            (26.45, 1.25466600674e-18)
            (26.55, 8.99339283327e-19)
            (26.65, 6.43831537983e-19)
            (26.75, 4.60335194823e-19)
            (26.85, 3.28722449091e-19)
            (26.95, 2.34443285122e-19)
            (27.05, 1.66993467712e-19)
            (27.15, 1.18799446872e-19)
            (27.25, 8.4407804572e-20)
            (27.35, 5.98968579583e-20)
            (27.45, 4.24501010056e-20)
            (27.55, 3.00473838758e-20)
            (27.65, 2.12416306694e-20)
            (27.75, 1.49976181505e-20)
            (27.85, 1.05757209009e-20)
            (27.95, 7.44819298106e-21)
            (28.05, 5.23896033906e-21)
            (28.15, 3.68037906197e-21)
            (28.25, 2.58222001125e-21)
            (28.35, 1.80945248816e-21)
            (28.45, 1.26635181234e-21)
            (28.55, 8.85145812128e-22)
            (28.65, 6.17914678758e-22)
            (28.75, 4.30819602682e-22)
            (28.85, 2.99996128501e-22)
            (28.95, 2.08635926551e-22)
            (29.05, 1.44915817184e-22)
            (29.15, 1.00530012366e-22)
            (29.25, 6.96512527879e-23)
            (29.35, 4.81964864566e-23)
            (29.45, 3.33085000925e-23)
            (29.55, 2.29904792455e-23)
            (29.65, 1.58487210959e-23)
            (29.75, 1.09117315194e-23)
            (29.85, 7.50319737954e-24)
            (29.95, 5.15290719914e-24)
            (30.05, 3.53436600746e-24)
            (30.15, 2.42116268885e-24)
            (30.25, 1.65649326383e-24)
            (30.35, 1.13190147588e-24)
            (30.45, 7.72468589143e-25)
            (30.55, 5.26509585005e-25)
            (30.65, 3.58414016634e-25)
            (30.75, 2.43678346783e-25)
            (30.85, 1.65463476232e-25)
            (30.95, 1.12212334942e-25)
            (31.05, 7.60032747389e-26)
            (31.15, 5.14135108393e-26)
            (31.25, 3.47356530261e-26)
            (31.35, 2.34383440641e-26)
            (31.45, 1.57954365373e-26)
            (31.55, 1.06313787694e-26)
            (31.65, 7.14662180414e-27)
            (31.75, 4.7980554132e-27)
            (31.85, 3.21723636678e-27)
            (31.95, 2.15453686561e-27)
            (32.05, 1.44104698277e-27)
            (32.15, 9.6262159862e-28)
            (32.25, 6.42223678486e-28)
            (32.35, 4.27927578485e-28)
            (32.45, 2.84778690817e-28)
            (32.55, 1.89277047403e-28)
            (32.65, 1.25643974683e-28)
            (32.75, 8.32987781583e-29)
            (32.85, 5.51555025699e-29)
            (32.95, 3.64747489785e-29)
            (33.05, 2.40906782773e-29)
            (33.15, 1.58912841908e-29)
            (33.25, 1.04694100884e-29)
            (33.35, 6.88872232287e-30)
            (33.45, 4.52697813937e-30)
            (33.55, 2.97119643961e-30)
            (33.65, 1.94763506608e-30)
            (33.75, 1.27507888762e-30)
            (33.85, 8.33719146235e-31)
            (33.95, 5.4444717145e-31)
            (34.05, 3.5509533522e-31)
            (34.15, 2.31306287906e-31)
            (34.25, 1.50481517198e-31)
            (34.35, 9.77759702149e-32)
            (34.45, 6.34503989975e-32)
            (34.55, 4.1123477384e-32)
            (34.65, 2.66194179846e-32)
            (34.75, 1.72091940065e-32)
            (34.85, 1.11115784298e-32)
            (34.95, 7.16546257404e-33)
            (35.05, 4.61493893123e-33)
            (35.15, 2.96852653565e-33)
            (35.25, 1.9070811036e-33)
            (35.35, 1.22363147874e-33)
            (35.45, 7.84125124538e-34)
            (35.55, 5.018493299e-34)
            (35.65, 3.20785388394e-34)
            (35.75, 2.04790147256e-34)
            (35.85, 1.30574024182e-34)
            (35.95, 8.3149141594e-35)
            (36.05, 5.28825028803e-35)
            (36.15, 3.35907332327e-35)
            (36.25, 2.13098413586e-35)
            (36.35, 1.35018786329e-35)
            (36.45, 8.54400381465e-36)
            (36.55, 5.39985275938e-36)
            (36.65, 3.4084399943e-36)
            (36.75, 2.1487339114e-36)
            (36.85, 1.35289120585e-36)
            (36.95, 8.50739034672e-37)
            (37.05, 5.34297444468e-37)
            (37.15, 3.35137545288e-37)
            (37.25, 2.09950214293e-37)
            (37.35, 1.31359900458e-37)
            (37.45, 8.20847631915e-38)
            (37.55, 5.12289599684e-38)
            (37.65, 3.19316807849e-38)
            (37.75, 1.98783928681e-38)
            (37.85, 1.23593038843e-38)
            (37.95, 7.67467515343e-39)
            (38.05, 4.75969631565e-39)
            (38.15, 2.94816466552e-39)
            (38.25, 1.82380112751e-39)
            (38.35, 1.12682499923e-39)
            (38.45, 6.95326397956e-40)
            (38.55, 4.28523074617e-40)
            (38.65, 2.63762443494e-40)
            (38.75, 1.62145519746e-40)
            (38.85, 9.95520481247e-41)
            (38.95, 6.10448029388e-41)
            (39.05, 3.73852632732e-41)
            (39.15, 2.28668025186e-41)
            (39.25, 1.39689476386e-41)
            (39.35, 8.52266049176e-42)
            (39.45, 5.19325846396e-42)
            (39.55, 3.16051559376e-42)
            (39.65, 1.92100806346e-42)
            (39.75, 1.16614804548e-42)
            (39.85, 7.07019578275e-43)
            (39.95, 4.28116969037e-43)
            (40.05, 2.58908730062e-43)
            (40.15, 1.5638107617e-43)
            (40.25, 9.43354555439e-44)
            (40.35, 5.68354044695e-44)
            (40.45, 3.41992206735e-44)
            (40.55, 2.0552597379e-44)
            (40.65, 1.23358894907e-44)
            (40.75, 7.3948177713e-45)
            (40.85, 4.42728757823e-45)
            (40.95, 2.64728829852e-45)
            (41.05, 1.58094950031e-45)
            (41.15, 9.42948569096e-46)
            (41.25, 5.61708847481e-46)
            (41.35, 3.3418563188e-46)
            (41.45, 1.98571782103e-46)
            (41.55, 1.17842119121e-46)
            (41.65, 6.98452384669e-47)
            (41.75, 4.13453160178e-47)
            (41.85, 2.44438200322e-47)
            (41.95, 1.44332817629e-47)
            (42.05, 8.51166167421e-48)
            (42.15, 5.01322118331e-48)
            (42.25, 2.94898545138e-48)
            (42.35, 1.73253351431e-48)
            (42.45, 1.01658549637e-48)
            (42.55, 5.95743653915e-49)
            (42.65, 3.48680926049e-49)
            (42.75, 2.03821600351e-49)
            (42.85, 1.189941125e-49)
            (42.95, 6.93831466138e-50)
            (43.05, 4.04050603526e-50)
            (43.15, 2.35001577927e-50)
            (43.25, 1.36508296615e-50)
            (43.35, 7.91955113044e-51)
            (43.45, 4.58875992854e-51)
            (43.55, 2.6554819749e-51)
            (43.65, 1.53477469128e-51)
            (43.75, 8.85929474401e-52)
            (43.85, 5.10748291878e-52)
            (43.95, 2.94081662996e-52)
            (44.05, 1.69115034616e-52)
            (44.15, 9.71291845448e-53)
            (44.25, 5.5714792433e-53)
            (44.35, 3.19186519924e-53)
            (44.45, 1.82629872956e-53)
            (44.55, 1.04364390913e-53)
            (44.65, 5.9564310545e-54)
            (44.75, 3.39526085274e-54)
            (44.85, 1.93291796928e-54)
            (44.95, 1.09902343048e-54)
            (45.05, 6.24099349395e-55)
            (45.15, 3.53959651472e-55)
            (45.25, 2.00496611994e-55)
            (45.35, 1.13426248492e-55)
            (45.45, 6.40875027074e-56)
            (45.55, 3.61648259932e-56)
            (45.65, 2.03822752133e-56)
            (45.75, 1.14728731933e-56)
            (45.85, 6.44978111126e-57)
            (45.95, 3.62135423753e-57)
            (46.05, 2.0307211537e-57)
            (46.15, 1.1373203944e-57)
            (46.25, 6.36163300621e-58)
            (46.35, 3.55391999498e-58)
            (46.45, 1.98289598113e-58)
            (46.55, 1.10495723216e-58)
            (46.65, 6.14956300696e-59)
            (46.75, 3.41819073604e-59)
            (46.85, 1.89758649397e-59)
            (46.95, 1.05210750588e-59)
            (47.05, 5.82601914932e-60)
            (47.15, 3.22208465873e-60)
            (47.25, 1.77973457254e-60)
            (47.35, 9.81808476706e-61)
            (47.45, 5.40943078537e-61)
            (47.55, 2.97666259231e-61)
            (47.65, 1.63591562952e-61)
            (47.75, 8.97936125138e-62)
            (47.85, 4.92247180358e-62)
            (47.95, 2.69509625547e-62)
            (48.05, 1.47373220687e-62)
            (48.15, 8.04852169835e-63)
            (48.25, 4.39002419997e-63)
            (48.35, 2.39150316452e-63)
            (48.45, 1.30115266625e-63)
            (48.55, 7.07031564306e-64)
            (48.65, 3.83709542443e-64)
    };

    \node[,
          below left=2pt
        ]
        at (rel axis cs:1,1)
        { $\mu$: 0.05, $\sigma$: 2.82 };

    \end{axis}
\end{tikzpicture}
\end{sansmath}
    \caption{\captitle{Detector offset distribution.} In black the mean
             detector offsets for all detectors relative to detector 2
             in each station are plotted. The gray line is a fit
             Gaussian. The mean of the offset distribution is
             \SI{0}{\nano\second} and the sigma is
             \SI{2.7}{\nano\second}.}
    \label{fig:detector_offset_distribution}
\end{figure}


\todo{Pulseheight/integral, Number of particles. Leptons and gammas,
      separate?}
