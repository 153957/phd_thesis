\chapter{Detector response}

\section{Signal response}

A particle is detected if it passes through our detector and leaves some
energy behind. These detection end up as a digitized list of numbers
with a timestamp. However, before the particle detection ends up in its
digital form it has been through some detector effects[? response].
These effects, which prevent the detector to have perfect timing and
particle counting will be outlined. These also serve as probability
distributions that are used in detector simulations.


\subsection{Energy loss}

Particles loose energy in the detector because of interactions with the
polyvinyltoluene in the scintillator. Ionization and Bremsstrahlung and
the main sources of energy loss for energetic charged particles
(electrons, muons). The scintillator is \SI{2}{\centi\meter} thick, if
particles enter de detector they may have a longer path in the detector.
It can also be shorter if the particle passes through one of the edges
of the scintillator, however, due to the large area of the detector the
chance of that happening is relatively low. The amount of energy
deposited by the particle going through the detector is described by the
Bethe-Bloch formula. This formula gives the stopping power, the amount
of energy lost by the particle per \SI{}{\gram\centi\meter\square} of
the material it passes through. However, the interactions in which the
particles loose energy is a statistical process, so the energy loss is
not a fixed number but a distribution. (fig...)

Energetic photons can also deposit energy in the scintillator due to
Compton scattering and pair creation. However, the cross sections for
these interactions is very low. A \SI{1}{\mega\electronvolt} photon  has
a mean free path greater than \SI{10}{\centi\meter}, which increases
further for higher energy photons, see fig [By Jos/Tom]. Despite the low
detection chance, the large number of photons in an air shower makes
them statistically relevant.

Temperature dependence, According to specifications the amount of light
produced should be constant between \SI{-60}{\degreeCelsius} and
\SI{20}{\degreeCelsius}. At \SI{60}{\degreeCelsius} the light output is
decreased by \SI{5}{\percent} relative to the output at
\SI{20}{\degreeCelsius}.

\subsection{Light transmission}

Because of the geometry of the detector the position of the scintillator
that the particle goes through is significant. The light emitted along
the path of the particle is transmitted in random directions. Depending
on the angle at which the emitted photons hit the outer edges of the
scintillator there is a chance that they will be reflected back or leave
the scintillator. The entire detector is wrapped in aluminium foil in
order to reflect those photons back into the scintillator. In order to
detect the photons they need to hit the windows of \pmt. To get to the
\pmt they need to pass from the scintillator through a layer of optical
glue, the light guide, another layer of optical glue, a small piece of
light guide, and finally a layer of optical tape to reach the \pmt.
During this many photons will be reflected a number of times. For
\SI{1}{\mip} approximately 30000 photons [check.] are emitted, most will
not make it to the \pmt.

A 2D simulation performed by Jos Steijger which tracked photons from
emission until they either reach the \pmt or leave the detector
(ignoring the aluminium foil).

Coincidence tests with a small (\SI[product-units = repeat]{1 x
1}{\centi\meter}) scintillator and \pmt and a normal \hisparc detector
also provide information about the transmission efficiency. By moving
the tiny detector over the large detector they will also trigger on
single muons passing through both detectors. This makes it possible to
check the efficiency at a specific point of the detector.

In the simulations we use the Bethe Bloch + efficiency modelled by Jos
as distribution which can be scaled by cosine of angle of incidence.

\begin{figure}
    \centering
    \includegraphics[width=0.6\textwidth]{plots/response/signal_efficiency.pdf}
    \caption{\captitle{Signal for a particle.} This distribution gives
             the signal from a single particles passing straight through
             a random location of the scintillator. The distribution
             scales with $\cos{\theta}^{-1}$.}
    \label{fig:signal_efficiency}
\end{figure}


Fokkema Fig 2.7, *2.8*, 2.9, 2.10.

Low efficiency: around ~2 \% across the detector, but not 0 \%!


\subsection{\pmt linearity}

Ideally the output of the \pmt grows is linear with the number of
particles in the detector. This is less important for a large number of
particles because the error on the number of particles also grows with
$\sqrt{n}$. Linearity is [currently] assumed in the analysis.

\begin{figure}
    \centering
    \includegraphics[width=0.6\textwidth]{plots/response/ph_vs_pi.pdf}
    \caption{\captitle{Pulseheight versus number of particles.}
             Indication of linearity of the \pmt.}
    \label{fig:transport_time}
\end{figure}


\subsection{\adcs}

The signal is sampled by two \adcs. In order to get good data and
triggering the \adcs need to be calibrated to have the same baseline and
response (gain). The \adcs normally have a range of
\SIrange{-1}{1}{\volt} translated into a 12-bit value. The \pmt outputs
negative pulses, so an electric circuit in the \hisparc electronics
modifies the incoming signal to make the effective \adc range
\SIrange{-2}{0.1}{\volt}. The baseline of the signals is calibrated to
be \SI{0}{\volt} at \SI{200}{\adc} and .. V at .. \Adcs to result in a
conversion factor of \SI{-0.57}{\milli\volt\per\adc}. In a 2-detector
station the \hisparc electronics triggers when both detectors cross the
low threshold of \SI{30}{\milli\volt} within the trigger window. 30 mV /
.57 = 53 above baseline. So the low threshold is set to a value of
\SI{253}{\adc}. Either of the two \adcs can be the one to cross this
threshold, so proper \adc alignments is necessary to ensure they treat
the signals equally.

\adcs are stable and accurate when they have a constant temperature.
When \hisparc electronics are first connected the temperature of the
\adcs will be below operating temperature. When calibration is performed
immediately it will drift from proper calibration as it continues to
operate. So calibration needs to be performed again.. (how much drift?)


\subsection{\mip/MPV}

The number of particles in a detector are determined by the pulse
integral, the integrated signal of the \pmt recorded by the \adc. When
making a histogram of all pulsintegrals this will not show discrete
values for the number of particles in each detection but a continuous
distribution. There is overlap in the pulse integrals from one particle
and the pulse integrals from two particles. The pulse integral histogram
does peak at a certain value, this is the most probable value for the
pulse for one particle, since it dominates the histogram. The value at
which this peak occurs is called the Most Probable Value (MPV).

A linear anti-correlation of several \SI{1}{\adc\per\kelvin} was found
between the MPV and the temperature of the \pmt (Bartels, Loran). Also
noted in the specifications as $\pm$
\SI{.5}{\percent\per\degreeCelsius}. The temperature of the \pmt can be
determined from the measured temperature outside and the solar
radiation. Best to determine the MPV in a period of constant
temperature, but with enough events for a good fit.


\section{Timing}


\subsection{Fluorescence}

deexcitation fast/slow process


\subsection{Light transport and \pmt efficiency}

Similar to the model that predicts the signal transport efficiency in
the scintillator a models that calculates the transport time of these
photons has been made. Quantum efficiency of the \pmt causes not every
photon reaching the \pmt to produce a photoelectron. Multiple photons
are required for the \pmt to produce a significant signal that the
\hisparc electronics triggers. The model tracks the time at which the
15th photon producing a photoelectron reaches the \pmt. 15
photoelectrons are required for a signal of \SI{70}{\milli\volt} (high
threshold).

In the simulations for each particle within the bounds of the
scintillator a number is taken from this distribution simulate the
transport time of each particle.

\begin{figure}
    \centering
    % \usepackage{tikz}
% \usetikzlibrary{arrows}
% \usepackage{pgfplots}
% \pgfplotsset{compat=1.3}
% \usepackage[detect-family]{siunitx}
% \usepackage[eulergreek]{sansmath}
% \sisetup{text-sf=\sansmath}
% \usepackage{relsize}
%
\tikzsetnextfilename{externalized-transport_time}

\pgfkeysifdefined{/artist/width}
    {\pgfkeysgetvalue{/artist/width}{\defaultwidth}}
    {\def\defaultwidth{ .67\linewidth }}
%
%
\begin{sansmath}
\begin{tikzpicture}[
        font=\sffamily,
        every pin/.style={inner sep=2pt, font={\sffamily\smaller}},
        every label/.style={inner sep=2pt, font={\sffamily\smaller}},
        every pin edge/.style={<-, >=stealth', shorten <=2pt},
        pin distance=2.5ex,
    ]
    \begin{axis}[
            axis background/.style={  },
            xmode=normal,
            ymode=normal,
            width=\defaultwidth,
            axis equal=false,
            %
            title={  },
            %
            xlabel={ Transport time [\si{\nano\second}] },
            ylabel={ Counts },
            %
            xmin={  },
            xmax={  },
            ymin={ 0.0 },
            ymax={  },
            %
            xtick={  },
            ytick={  },
            %
            tick align=outside,
            max space between ticks=40,
            every tick/.style={},
            axis on top,
        ]

        






    
    % Draw series plot
    \addplot[no markers,gray,const plot] coordinates {
        (1.0, 0)
        (1.1, 0)
        (1.2, 1)
        (1.3, 3)
        (1.4, 2)
        (1.5, 4)
        (1.6, 3)
        (1.7, 2)
        (1.8, 5)
        (1.9, 19)
        (2.0, 22)
        (2.1, 26)
        (2.2, 34)
        (2.3, 58)
        (2.4, 78)
        (2.5, 123)
        (2.6, 201)
        (2.7, 296)
        (2.8, 414)
        (2.9, 445)
        (3.0, 435)
        (3.1, 460)
        (3.2, 552)
        (3.3, 579)
        (3.4, 304)
        (3.5, 196)
        (3.6, 198)
        (3.7, 185)
        (3.8, 191)
        (3.9, 193)
        (4.0, 165)
        (4.1, 203)
        (4.2, 177)
        (4.3, 218)
        (4.4, 239)
        (4.5, 199)
        (4.6, 205)
        (4.7, 192)
        (4.8, 218)
        (4.9, 165)
        (5.0, 230)
        (5.1, 183)
        (5.2, 216)
        (5.3, 177)
        (5.4, 226)
        (5.5, 208)
        (5.6, 191)
        (5.7, 188)
        (5.8, 153)
        (5.9, 178)
        (6.0, 165)
        (6.1, 197)
        (6.2, 149)
        (6.3, 130)
        (6.4, 117)
        (6.5, 89)
        (6.6, 75)
        (6.7, 55)
        (6.8, 35)
        (6.9, 20)
        (7.0, 5)
        (7.1, 2)
        (7.2, 1)
        (7.3, 0)
        (7.4, 0)
        (7.5, 0)
        (7.6, 0)
        (7.7, 0)
        (7.8, 0)
        (7.9, 0)
        (8.0, 0)
    };
    
    % Draw series plot
    \addplot[no markers,dashed, lightgray] coordinates {
        (2.5507, 0)
        (2.5507, 380.96)
        (3.4953, 380.96)
        (3.4953, 186.68)
        (6.463, 186.68)
        (6.463, 0)
    };
    
    % Draw series plot
    \addplot[no markers,solid,const plot] coordinates {
        (1.0, 0)
        (1.1, 0)
        (1.2, 0)
        (1.3, 0)
        (1.4, 0)
        (1.5, 0)
        (1.6, 0)
        (1.7, 0)
        (1.8, 0)
        (1.9, 0)
        (2.0, 0)
        (2.1, 0)
        (2.2, 0)
        (2.3, 0)
        (2.4, 0)
        (2.5, 221)
        (2.6, 437)
        (2.7, 415)
        (2.8, 440)
        (2.9, 412)
        (3.0, 402)
        (3.1, 407)
        (3.2, 404)
        (3.3, 414)
        (3.4, 378)
        (3.5, 210)
        (3.6, 226)
        (3.7, 205)
        (3.8, 216)
        (3.9, 190)
        (4.0, 199)
        (4.1, 199)
        (4.2, 202)
        (4.3, 216)
        (4.4, 179)
        (4.5, 192)
        (4.6, 219)
        (4.7, 242)
        (4.8, 216)
        (4.9, 198)
        (5.0, 207)
        (5.1, 217)
        (5.2, 195)
        (5.3, 220)
        (5.4, 196)
        (5.5, 213)
        (5.6, 183)
        (5.7, 215)
        (5.8, 206)
        (5.9, 219)
        (6.0, 183)
        (6.1, 185)
        (6.2, 205)
        (6.3, 190)
        (6.4, 127)
        (6.5, 0)
        (6.6, 0)
        (6.7, 0)
        (6.8, 0)
        (6.9, 0)
        (7.0, 0)
        (7.1, 0)
        (7.2, 0)
        (7.3, 0)
        (7.4, 0)
        (7.5, 0)
        (7.6, 0)
        (7.7, 0)
        (7.8, 0)
        (7.9, 0)
        (8.0, 0)
    };







    \end{axis}
\end{tikzpicture}
\end{sansmath}

    \caption{\captitle{Signal transport time.} Time transport
             distribution in the detector.}
    \label{fig:transport_time}
\end{figure}


\subsection{\pmt transport, cable length}

The time it takes for the kaskade in the \pmt to go from the front
window to the anode is dependent on .. high voltage of pmt, position
electron hits, etc? Possible second pulse from ionization of atoms in
tube.

The signal from the \pmt is transported over a \SI{30}{\meter} long
coaxial cable to the \hisparc electronics before it is digitized. In
order to keep the detectors in sync the cables are precisely cut to the
same length. The signal in the cables propagates at 2/3 the speed of
light or \SI{.2}{\meter\per\nano\second}. Every \SI{.2}{\meter}
difference in cable length increases the offset by a nanosecond. Kinks
and twists in the cable can also increase the propagation time.

The offset between detectors can change over time due to aging of the
\pmt, changes in the high voltage and modifications to the cables.

Offsets are determined by looking at the arrival time differences
between two detectors over many events. All events in which a signal was
detected in the reference detector (usually detector 2, because of its
central location in 4-detector stations) and in another detector are
used. Multiple events are needed to even out time differences due to the
arrival direction of the showers, transport time and the shower front.
The time difference distribution between two detectors includes a
background of accidental coincidences.

To determine the distribution of the mean offsets the distributions for
a day of data for all stations have been fit by Gaussians, the found
averages are then the distribution of the offsets. These have also been
fit by a Gaussian \figref{fig:detector_offset_distribution}. In
simulations a detector offset is drawn from this distribution and then
fixed for that detector for the entire simulation.

[relative to detector 2.. different sigma if offset also given to detector 2?]

\begin{figure}
    \centering
    \includegraphics[width=0.6\textwidth]{plots/response/detector_offset_distribution_20140101_d2.pdf}
    \caption{\captitle{Detector offset distribution.} In black the mean
             detector offsets for all detectors relative to detector 2
             in each station are plotted. The gray line is a fit
             Gaussian. The mean of the offset distribution is
             \SI{0}{\nano\second} and the sigma is
             \SI{2.7}{\nano\second}.}
    \label{fig:detector_offset_distribution}
\end{figure}


\subsection{\adc sampling}

The \pmt signals are sampled every \SI{2.5}{\nano\second} by one of the
two \adcs. So the exact start of the signal is not known, can perhaps be
derived from fit to the rising flank. At least it is simulated. sampling
is sync in a station, not between stations, hence the \gps.
