\chapter{Cluster response}


\section{Positions}

\subsection{\gps Self-Survey}

Two \gps antennae separated by \SI{2}{\meter} are used by station 501
and station 510. Each station performed a self-survey with its own \gps
and one with the \gps of the other station to test the accuracy.
 
GPS s510
s510  1412347130	52.3558955473	4.95098636604	54.5890849568
s501  1414406502	52.355899096	4.95098802631	57.646343655

GPS s501
s501  1412347557	52.355880072	4.95100645695	55.4043086618
s510  1414406952	52.3558859189	4.95100828296	57.3941961704

Difference of \SI{1.}{\centi\meter}.


\section{Timing}

\subsection{\gps timestamp, electronics}

When considering coincidences between stations the accuracy of the
timing is crucial. Similar to the offsets between detectors in a
station, offsets between stations have been found. These can be caused
by different detector cable lengths, \gps antennae cable length, and
possibly something in the electronics of the \hisparc box.

Determining the station offset requires many coincidences. The number of
coincidences between stations is inversely correlated to the the
distance between them. Just as the time difference distribution between
detectors the station time difference is dominated by time differences
due to the arrival direction of the showers. This time difference
increases with distance $\Delta t = r * \sin{\theta}$.

\begin{figure}
    \centering
    \includegraphics[width=0.6\textwidth]
        {plots/response/station_offsets_501_502.pdf}
    \caption{\captitle{Coincidence time difference distribution.} The
             time difference between events in coincidence detected by
             station 501 and 502 in the period 2010-2014. The width ..
             due to distance between stations and showers under angles.
             The mean is \SI{-26.6}{\nano\second}.}
    \label{fig:station_offsets_501_502}
\end{figure}
