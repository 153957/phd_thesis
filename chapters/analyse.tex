\chapter{Analyse}
\label{ch:analyse}


\section{Measured data}

Raw data consists of events with traces. An event is when the trigger
conditions are met in the \hisparc box of a station. A trigger requires
2 low signals in a 2-detector station or 2 high or 3 low signals in a
4-detector station. When a trigger occurs all \pmt channels of that
station are read-out, even if they do not have significant signals.
These traces are then processed by the \daq or \pysparc. Derived
values, such as the baseline, standard deviation, number of peaks,
pulseheight and pulseintegral are determined for each trace. The traces
are also reduced by cutting parts from the front and end until the first
significant signal in any detector, with some buffer room?

During the trigger the \gps and \SI{400}{\mega\hertz} clock are used to
determine the \gps timestamp of the event. Besides this event data there
is also static data.

When a station is deployed, moved or changed a \gps self-survey is run,
this accurately determines the position of the \gps antenna. The
detector positions relative to the \gps antenna are also measured. If
they are moved the positions are again measured.

Finally the settings, e.g. \pmt voltages, trigger settings and other
settings, of a station are stored. Currently only when they change and
the button is pressed... To be sent every hour..

Config should be used for esd....


\section{Reconstruction chain}

Raw data is received by the frome server and stored in the \hisparc
datastore. This is done by the datastore software. Multiple listeners
receive data from stations and place that data in a temporary storage
location. A single writer process checks the temporary storage for new
data and writes it to the raw datastorage.

Every night all data from the previous day is processed by the pique
server and stored in the Event Summary Data. This data processing
consists of several steps.

First all events are sorted by their timestamp. Then the duplicate
events are removed. A station number and extended timestamp (timestamp
in nanoseconds) should be a unique combination. Due to server crashes
and errors in uploading duplicate events may exist.

Next the MPV of the MIP peak for each detector is determined by fitting
the pulse integral histogram. Using this the number of MIPs in each
event is derived by dividing the pulse integral by the MPV. The MPV can
change when the voltages on the PMTs change or when the temperature of
the PMT changes. Higher (voltage/temp) -> higher (MPV). After that the
traces of each event are analysed to determine the moment of the trigger
and the arrival time (first pulse) in each detector.

In the ESD the sorted events with duplicates removed and MPV determined,
MIP derived, trigger reconstructed and arrival times determined data is
stored. Station data is stored with the same structure as the raw
storage. When data for all stations is processed the next step is to see
if links exist between events. So coincidences between all stations are
searched. A coincidence is more than one event detected in a small time
window by multiple stations.

speed of light, distance, coincidence window, clusters, up time..

After this some histograms are made from the data and made available on
\url{http://data.hisparc.nl} to get an overview of the data.

Diagram different steps in reconstruction/checks.


\section{Quality assurance}

When doing reconstructions it is important to know if the data that is
being used is actually good data. If for instance a \gps is not properly
configured the larger timing uncertainty due to the misconfiguration
needs to be taken into account. Additionally the uptime of
stations/detectors is important. A station that is working but does not
make a detection needs to be considered differently from a station that
have no data because it is offline.

Several parameters are considered for data quality, namely;

- Determine from number of events
- MIP peak location
- Expected number of coincidences
- Correctness of configuration
- Changes in configuration
- \gps settings/location accuracy
- ...

Why can data be low quality?

Technical issues:
- power supply issues
- electronics
- ...

Software issues:
- firewall
- software stability
- self-healing/self-help
- ...


\section{Data cuts}

Good data quality.
Known detector positions.


\section{Event reconstruction}

Event reconstruction consists of several steps. We are interested in the
shower direction, core location and primary particle energy. Knowledge
of one  parameter can improve the reconstruction of another paramater.
So the reconstruction is an iterative process in which each type of
reconstruction is performed multiple times. Each time better knowledge
about other paramaters is known.


\section{Direction reconstruction}

\subsection{Algorithms}

3 detection points can be solved analytical for flat front.
More than 3 can be done with regression.

\subsection{Single station}

\begin{figure}
    \centering
    % \usepackage{tikz}
% \usetikzlibrary{arrows}
% \usepackage{pgfplots}
% \pgfplotsset{compat=1.3}
% \usepackage[detect-family]{siunitx}
% \usepackage[eulergreek]{sansmath}
% \sisetup{text-sf=\sansmath}
% \usepackage{relsize}
%
\usepgfplotslibrary{polar}

\pgfkeysifdefined{/artist/width}
    {\pgfkeysgetvalue{/artist/width}{\defaultwidth}}
    {\def\defaultwidth{ .67\linewidth }}
%
%
\begin{sansmath}
\begin{tikzpicture}[
        font=\sffamily,
        every pin/.style={inner sep=2pt, font={\sffamily\smaller}},
        every label/.style={inner sep=2pt, font={\sffamily\smaller}},
        every pin edge/.style={<-, >=stealth', shorten <=2pt},
        pin distance=2.5ex,
    ]
    \begin{polaraxis}[
            axis background/.style={  },
            grid style={opacity=.3},
            xmode=normal,
            ymode=normal,
            width=\defaultwidth,
            axis equal=false,
            y axis line style={yshift=-5cm},
            major y tick style={yshift=-5cm},
            %
            title={  },
            %
            xlabel={ Azimuth (degrees) },
            ylabel={ Zenith (degrees) },
            xlabel style={xshift=1.5cm, rotate=-30},
            ylabel style={xshift=-.5cm, rotate=-90},
            %
            xmin={  },
            xmax={  },
            ymin={ 0 },
            ymax={ 90 },
            %
            xtick={  },
            ytick={  },
            xticklabel=$\pgfmathprintnumber{\tick}^\circ$,
            yticklabel=$\pgfmathprintnumber{\tick}^\circ$,
            yticklabel shift={-5.75cm},
            %
            ytick align=outside,
            xtick align=inside,
            max space between ticks=40,
            every tick/.style={},
            axis on top,
        ]

        




    
    % Draw series plot
    \addplot[mark=*,only marks,mark options={mark size=.75pt}] coordinates {
        (-107.480170203, 87.1340160174)
        (-112.410910531, 79.6140724773)
        (-117.457076096, 77.4116415297)
        (-122.542923904, 77.4116415297)
        (-127.589089469, 79.6140724773)
        (-132.519829797, 87.1340160174)
        (-94.3066190955, 87.1340160174)
        (-98.9482755646, 74.6582914455)
        (-103.897886248, 69.5126848853)
        (-109.106605351, 66.4218215218)
        (-114.503633455, 64.7099922335)
        (-120.0, 64.1580672368)
        (-125.496366545, 64.7099922335)
        (-130.893394649, 66.4218215218)
        (-136.102113752, 69.5126848853)
        (-141.051724435, 74.6582914455)
        (-145.693380904, 87.1340160174)
        (-85.6933809045, 87.1340160174)
        (-90.0, 72.2936809445)
        (-94.7150039539, 65.8411626967)
        (-99.8264298158, 61.5123318158)
        (-105.295344301, 58.5324671144)
        (-111.051724435, 56.6329870308)
        (-116.995508401, 55.7037913519)
        (-123.004491599, 55.7037913519)
        (-128.948275565, 56.6329870308)
        (-134.704655699, 58.5324671144)
        (-140.173570184, 61.5123318158)
        (-145.284996046, 65.8411626967)
        (-150.0, 72.2936809445)
        (-154.306619096, 87.1340160174)
        (-81.0517244354, 74.6582914455)
        (-85.2849960461, 65.8411626967)
        (-90.0, 60.0)
        (-95.2087191029, 55.7037913519)
        (-100.893394649, 52.5350310724)
        (-106.996088057, 50.3305661989)
        (-113.413224446, 49.0239231237)
        (-120.0, 48.5903778907)
        (-126.586775554, 49.0239231237)
        (-133.003911943, 50.3305661989)
        (-139.106605351, 52.5350310724)
        (-144.791280897, 55.7037913519)
        (-150.0, 60.0)
        (-154.715003954, 65.8411626967)
        (-158.948275565, 74.6582914455)
        (-72.5198297972, 87.1340160174)
        (-76.102113752, 69.5126848853)
        (-80.1735701842, 61.5123318158)
        (-84.7912808971, 55.7037913519)
        (-90.0, 51.2077636525)
        (-95.8175256444, 47.7256599062)
        (-102.21634884, 45.1432400456)
        (-109.106605351, 43.423570598)
        (-116.329503492, 42.5619875652)
        (-123.670496508, 42.5619875652)
        (-130.893394649, 43.423570598)
        (-137.78365116, 45.1432400456)
        (-144.182474356, 47.7256599062)
        (-150.0, 51.2077636525)
        (-155.208719103, 55.7037913519)
        (-159.826429816, 61.5123318158)
        (-163.897886248, 69.5126848853)
        (-167.480170203, 87.1340160174)
        (-67.589089469, 79.6140724773)
        (-70.8933946491, 66.4218215218)
        (-74.7046556986, 58.5324671144)
        (-79.1066053509, 52.5350310724)
        (-84.1824743556, 47.7256599062)
        (-90.0, 43.853778612)
        (-96.5867755536, 40.8313604055)
        (-103.897886248, 38.6454835028)
        (-111.786789298, 37.3165236011)
        (-120.0, 36.8698976458)
        (-128.213210702, 37.3165236011)
        (-136.102113752, 38.6454835028)
        (-143.413224446, 40.8313604055)
        (-150.0, 43.853778612)
        (-155.817525644, 47.7256599062)
        (-160.893394649, 52.5350310724)
        (-165.295344301, 58.5324671144)
        (-169.106605351, 66.4218215218)
        (-172.410910531, 79.6140724773)
        (-62.5429239041, 77.4116415297)
        (-65.4963665447, 64.7099922335)
        (-68.9482755646, 56.6329870308)
        (-73.0039119428, 50.3305661989)
        (-77.7836511603, 45.1432400456)
        (-83.4132244464, 40.8313604055)
        (-90.0, 37.3165236011)
        (-97.589089469, 34.6032778097)
        (-106.102113752, 32.7403417583)
        (-115.284996046, 31.7883306171)
        (-124.715003954, 31.7883306171)
        (-133.897886248, 32.7403417583)
        (-142.410910531, 34.6032778097)
        (-150.0, 37.3165236011)
        (-156.586775554, 40.8313604055)
        (-162.21634884, 45.1432400456)
        (-166.996088057, 50.3305661989)
        (-171.051724435, 56.6329870308)
        (-174.503633455, 64.7099922335)
        (-177.457076096, 77.4116415297)
        (-57.4570760959, 77.4116415297)
        (-60.0, 64.1580672368)
        (-63.0044915989, 55.7037913519)
        (-66.5867755536, 49.0239231237)
        (-70.8933946491, 43.423570598)
        (-76.102113752, 38.6454835028)
        (-82.410910531, 34.6032778097)
        (-90.0, 31.3064462487)
        (-98.9482755646, 28.8280442934)
        (-109.106605351, 27.2747286804)
        (-120.0, 26.7436839504)
        (-130.893394649, 27.2747286804)
        (-141.051724435, 28.8280442934)
        (-150.0, 31.3064462487)
        (-157.589089469, 34.6032778097)
        (-163.897886248, 38.6454835028)
        (-169.106605351, 43.423570598)
        (-173.413224446, 49.0239231237)
        (-176.995508401, 55.7037913519)
        (180.0, 64.1580672368)
        (177.457076096, 77.4116415297)
        (-52.410910531, 79.6140724773)
        (-54.5036334553, 64.7099922335)
        (-56.9955084011, 55.7037913519)
        (-60.0, 48.5903778907)
        (-63.6704965083, 42.5619875652)
        (-68.2132107017, 37.3165236011)
        (-73.897886248, 32.7403417583)
        (-81.0517244354, 28.8280442934)
        (-90.0, 25.6589062733)
        (-100.893394649, 23.3821964255)
        (-113.413224446, 22.1784004191)
        (-126.586775554, 22.1784004191)
        (-139.106605351, 23.3821964255)
        (-150.0, 25.6589062733)
        (-158.948275565, 28.8280442934)
        (-166.102113752, 32.7403417583)
        (-171.786789298, 37.3165236011)
        (-176.329503492, 42.5619875652)
        (180.0, 48.5903778907)
        (176.995508401, 55.7037913519)
        (174.503633455, 64.7099922335)
        (172.410910531, 79.6140724773)
        (-47.4801702028, 87.1340160174)
        (-49.1066053509, 66.4218215218)
        (-51.0517244354, 56.6329870308)
        (-53.4132244464, 49.0239231237)
        (-56.3295034917, 42.5619875652)
        (-60.0, 36.8698976458)
        (-64.7150039539, 31.7883306171)
        (-70.8933946491, 27.2747286804)
        (-79.1066053509, 23.3821964255)
        (-90.0, 20.2679010557)
        (-103.897886248, 18.1948723388)
        (-120.0, 17.4576031237)
        (-136.102113752, 18.1948723388)
        (-150.0, 20.2679010557)
        (-160.893394649, 23.3821964255)
        (-169.106605351, 27.2747286804)
        (-175.284996046, 31.7883306171)
        (180.0, 36.8698976458)
        (176.329503492, 42.5619875652)
        (173.413224446, 49.0239231237)
        (171.051724435, 56.6329870308)
        (169.106605351, 66.4218215218)
        (167.480170203, 87.1340160174)
        (-43.897886248, 69.5126848853)
        (-45.2953443014, 58.5324671144)
        (-46.9960880572, 50.3305661989)
        (-49.1066053509, 43.423570598)
        (-51.7867892983, 37.3165236011)
        (-55.2849960461, 31.7883306171)
        (-60.0, 26.7436839504)
        (-66.5867755536, 22.1784004191)
        (-76.102113752, 18.1948723388)
        (-90.0, 15.0586473737)
        (-109.106605351, 13.2457851002)
        (-130.893394649, 13.2457851002)
        (-150.0, 15.0586473737)
        (-163.897886248, 18.1948723388)
        (-173.413224446, 22.1784004191)
        (180.0, 26.7436839504)
        (175.284996046, 31.7883306171)
        (171.786789298, 37.3165236011)
        (169.106605351, 43.423570598)
        (166.996088057, 50.3305661989)
        (165.295344301, 58.5324671144)
        (163.897886248, 69.5126848853)
        (-38.9482755646, 74.6582914455)
        (-39.8264298158, 61.5123318158)
        (-40.8933946491, 52.5350310724)
        (-42.2163488397, 45.1432400456)
        (-43.897886248, 38.6454835028)
        (-46.102113752, 32.7403417583)
        (-49.1066053509, 27.2747286804)
        (-53.4132244464, 22.1784004191)
        (-60.0, 17.4576031237)
        (-70.8933946491, 13.2457851002)
        (-90.0, 9.9742217944)
        (-120.0, 8.62692655868)
        (-150.0, 9.9742217944)
        (-169.106605351, 13.2457851002)
        (180.0, 17.4576031237)
        (173.413224446, 22.1784004191)
        (169.106605351, 27.2747286804)
        (166.102113752, 32.7403417583)
        (163.897886248, 38.6454835028)
        (162.21634884, 45.1432400456)
        (160.893394649, 52.5350310724)
        (159.826429816, 61.5123318158)
        (158.948275565, 74.6582914455)
        (-34.3066190955, 87.1340160174)
        (-34.7150039539, 65.8411626967)
        (-35.2087191029, 55.7037913519)
        (-35.8175256444, 47.7256599062)
        (-36.5867755536, 40.8313604055)
        (-37.589089469, 34.6032778097)
        (-38.9482755646, 28.8280442934)
        (-40.8933946491, 23.3821964255)
        (-43.897886248, 18.1948723388)
        (-49.1066053509, 13.2457851002)
        (-60.0, 8.62692655868)
        (-90.0, 4.96818353607)
        (-150.0, 4.96818353607)
        (180.0, 8.62692655868)
        (169.106605351, 13.2457851002)
        (163.897886248, 18.1948723388)
        (160.893394649, 23.3821964255)
        (158.948275565, 28.8280442934)
        (157.589089469, 34.6032778097)
        (156.586775554, 40.8313604055)
        (155.817525644, 47.7256599062)
        (155.208719103, 55.7037913519)
        (154.715003954, 65.8411626967)
        (154.306619096, 87.1340160174)
        (-30.0, 72.2936809445)
        (-30.0, 60.0)
        (-30.0, 51.2077636525)
        (-30.0, 43.853778612)
        (-30.0, 37.3165236011)
        (-30.0, 31.3064462487)
        (-30.0, 25.6589062733)
        (-30.0, 20.2679010557)
        (-30.0, 15.0586473737)
        (-30.0, 9.9742217944)
        (-30.0, 4.96818353607)
        (0.0, 0.0)
        (150.0, 4.96818353607)
        (150.0, 9.9742217944)
        (150.0, 15.0586473737)
        (150.0, 20.2679010557)
        (150.0, 25.6589062733)
        (150.0, 31.3064462487)
        (150.0, 37.3165236011)
        (150.0, 43.853778612)
        (150.0, 51.2077636525)
        (150.0, 60.0)
        (150.0, 72.2936809445)
        (-25.6933809045, 87.1340160174)
        (-25.2849960461, 65.8411626967)
        (-24.7912808971, 55.7037913519)
        (-24.1824743556, 47.7256599062)
        (-23.4132244464, 40.8313604055)
        (-22.410910531, 34.6032778097)
        (-21.0517244354, 28.8280442934)
        (-19.1066053509, 23.3821964255)
        (-16.102113752, 18.1948723388)
        (-10.8933946491, 13.2457851002)
        (0.0, 8.62692655868)
        (30.0, 4.96818353607)
        (90.0, 4.96818353607)
        (120.0, 8.62692655868)
        (130.893394649, 13.2457851002)
        (136.102113752, 18.1948723388)
        (139.106605351, 23.3821964255)
        (141.051724435, 28.8280442934)
        (142.410910531, 34.6032778097)
        (143.413224446, 40.8313604055)
        (144.182474356, 47.7256599062)
        (144.791280897, 55.7037913519)
        (145.284996046, 65.8411626967)
        (145.693380904, 87.1340160174)
        (-21.0517244354, 74.6582914455)
        (-20.1735701842, 61.5123318158)
        (-19.1066053509, 52.5350310724)
        (-17.7836511603, 45.1432400456)
        (-16.102113752, 38.6454835028)
        (-13.897886248, 32.7403417583)
        (-10.8933946491, 27.2747286804)
        (-6.58677555363, 22.1784004191)
        (0.0, 17.4576031237)
        (10.8933946491, 13.2457851002)
        (30.0, 9.9742217944)
        (60.0, 8.62692655868)
        (90.0, 9.9742217944)
        (109.106605351, 13.2457851002)
        (120.0, 17.4576031237)
        (126.586775554, 22.1784004191)
        (130.893394649, 27.2747286804)
        (133.897886248, 32.7403417583)
        (136.102113752, 38.6454835028)
        (137.78365116, 45.1432400456)
        (139.106605351, 52.5350310724)
        (140.173570184, 61.5123318158)
        (141.051724435, 74.6582914455)
        (-16.102113752, 69.5126848853)
        (-14.7046556986, 58.5324671144)
        (-13.0039119428, 50.3305661989)
        (-10.8933946491, 43.423570598)
        (-8.21321070174, 37.3165236011)
        (-4.71500395395, 31.7883306171)
        (0.0, 26.7436839504)
        (6.58677555363, 22.1784004191)
        (16.102113752, 18.1948723388)
        (30.0, 15.0586473737)
        (49.1066053509, 13.2457851002)
        (70.8933946491, 13.2457851002)
        (90.0, 15.0586473737)
        (103.897886248, 18.1948723388)
        (113.413224446, 22.1784004191)
        (120.0, 26.7436839504)
        (124.715003954, 31.7883306171)
        (128.213210702, 37.3165236011)
        (130.893394649, 43.423570598)
        (133.003911943, 50.3305661989)
        (134.704655699, 58.5324671144)
        (136.102113752, 69.5126848853)
        (-12.5198297972, 87.1340160174)
        (-10.8933946491, 66.4218215218)
        (-8.94827556463, 56.6329870308)
        (-6.58677555363, 49.0239231237)
        (-3.67049650832, 42.5619875652)
        (0.0, 36.8698976458)
        (4.71500395395, 31.7883306171)
        (10.8933946491, 27.2747286804)
        (19.1066053509, 23.3821964255)
        (30.0, 20.2679010557)
        (43.897886248, 18.1948723388)
        (60.0, 17.4576031237)
        (76.102113752, 18.1948723388)
        (90.0, 20.2679010557)
        (100.893394649, 23.3821964255)
        (109.106605351, 27.2747286804)
        (115.284996046, 31.7883306171)
        (120.0, 36.8698976458)
        (123.670496508, 42.5619875652)
        (126.586775554, 49.0239231237)
        (128.948275565, 56.6329870308)
        (130.893394649, 66.4218215218)
        (132.519829797, 87.1340160174)
        (-7.58908946897, 79.6140724773)
        (-5.49636654465, 64.7099922335)
        (-3.00449159888, 55.7037913519)
        (0.0, 48.5903778907)
        (3.67049650832, 42.5619875652)
        (8.21321070174, 37.3165236011)
        (13.897886248, 32.7403417583)
        (21.0517244354, 28.8280442934)
        (30.0, 25.6589062733)
        (40.8933946491, 23.3821964255)
        (53.4132244464, 22.1784004191)
        (66.5867755536, 22.1784004191)
        (79.1066053509, 23.3821964255)
        (90.0, 25.6589062733)
        (98.9482755646, 28.8280442934)
        (106.102113752, 32.7403417583)
        (111.786789298, 37.3165236011)
        (116.329503492, 42.5619875652)
        (120.0, 48.5903778907)
        (123.004491599, 55.7037913519)
        (125.496366545, 64.7099922335)
        (127.589089469, 79.6140724773)
        (-2.54292390406, 77.4116415297)
        (0.0, 64.1580672368)
        (3.00449159888, 55.7037913519)
        (6.58677555363, 49.0239231237)
        (10.8933946491, 43.423570598)
        (16.102113752, 38.6454835028)
        (22.410910531, 34.6032778097)
        (30.0, 31.3064462487)
        (38.9482755646, 28.8280442934)
        (49.1066053509, 27.2747286804)
        (60.0, 26.7436839504)
        (70.8933946491, 27.2747286804)
        (81.0517244354, 28.8280442934)
        (90.0, 31.3064462487)
        (97.589089469, 34.6032778097)
        (103.897886248, 38.6454835028)
        (109.106605351, 43.423570598)
        (113.413224446, 49.0239231237)
        (116.995508401, 55.7037913519)
        (120.0, 64.1580672368)
        (122.542923904, 77.4116415297)
        (2.54292390406, 77.4116415297)
        (5.49636654465, 64.7099922335)
        (8.94827556463, 56.6329870308)
        (13.0039119428, 50.3305661989)
        (17.7836511603, 45.1432400456)
        (23.4132244464, 40.8313604055)
        (30.0, 37.3165236011)
        (37.589089469, 34.6032778097)
        (46.102113752, 32.7403417583)
        (55.2849960461, 31.7883306171)
        (64.7150039539, 31.7883306171)
        (73.897886248, 32.7403417583)
        (82.410910531, 34.6032778097)
        (90.0, 37.3165236011)
        (96.5867755536, 40.8313604055)
        (102.21634884, 45.1432400456)
        (106.996088057, 50.3305661989)
        (111.051724435, 56.6329870308)
        (114.503633455, 64.7099922335)
        (117.457076096, 77.4116415297)
        (7.58908946897, 79.6140724773)
        (10.8933946491, 66.4218215218)
        (14.7046556986, 58.5324671144)
        (19.1066053509, 52.5350310724)
        (24.1824743556, 47.7256599062)
        (30.0, 43.853778612)
        (36.5867755536, 40.8313604055)
        (43.897886248, 38.6454835028)
        (51.7867892983, 37.3165236011)
        (60.0, 36.8698976458)
        (68.2132107017, 37.3165236011)
        (76.102113752, 38.6454835028)
        (83.4132244464, 40.8313604055)
        (90.0, 43.853778612)
        (95.8175256444, 47.7256599062)
        (100.893394649, 52.5350310724)
        (105.295344301, 58.5324671144)
        (109.106605351, 66.4218215218)
        (112.410910531, 79.6140724773)
        (12.5198297972, 87.1340160174)
        (16.102113752, 69.5126848853)
        (20.1735701842, 61.5123318158)
        (24.7912808971, 55.7037913519)
        (30.0, 51.2077636525)
        (35.8175256444, 47.7256599062)
        (42.2163488397, 45.1432400456)
        (49.1066053509, 43.423570598)
        (56.3295034917, 42.5619875652)
        (63.6704965083, 42.5619875652)
        (70.8933946491, 43.423570598)
        (77.7836511603, 45.1432400456)
        (84.1824743556, 47.7256599062)
        (90.0, 51.2077636525)
        (95.2087191029, 55.7037913519)
        (99.8264298158, 61.5123318158)
        (103.897886248, 69.5126848853)
        (107.480170203, 87.1340160174)
        (21.0517244354, 74.6582914455)
        (25.2849960461, 65.8411626967)
        (30.0, 60.0)
        (35.2087191029, 55.7037913519)
        (40.8933946491, 52.5350310724)
        (46.9960880572, 50.3305661989)
        (53.4132244464, 49.0239231237)
        (60.0, 48.5903778907)
        (66.5867755536, 49.0239231237)
        (73.0039119428, 50.3305661989)
        (79.1066053509, 52.5350310724)
        (84.7912808971, 55.7037913519)
        (90.0, 60.0)
        (94.7150039539, 65.8411626967)
        (98.9482755646, 74.6582914455)
        (25.6933809045, 87.1340160174)
        (30.0, 72.2936809445)
        (34.7150039539, 65.8411626967)
        (39.8264298158, 61.5123318158)
        (45.2953443014, 58.5324671144)
        (51.0517244354, 56.6329870308)
        (56.9955084011, 55.7037913519)
        (63.0044915989, 55.7037913519)
        (68.9482755646, 56.6329870308)
        (74.7046556986, 58.5324671144)
        (80.1735701842, 61.5123318158)
        (85.2849960461, 65.8411626967)
        (90.0, 72.2936809445)
        (94.3066190955, 87.1340160174)
        (34.3066190955, 87.1340160174)
        (38.9482755646, 74.6582914455)
        (43.897886248, 69.5126848853)
        (49.1066053509, 66.4218215218)
        (54.5036334553, 64.7099922335)
        (60.0, 64.1580672368)
        (65.4963665447, 64.7099922335)
        (70.8933946491, 66.4218215218)
        (76.102113752, 69.5126848853)
        (81.0517244354, 74.6582914455)
        (85.6933809045, 87.1340160174)
        (47.4801702028, 87.1340160174)
        (52.410910531, 79.6140724773)
        (57.4570760959, 77.4116415297)
        (62.5429239041, 77.4116415297)
        (67.589089469, 79.6140724773)
        (72.5198297972, 87.1340160174)
    };

    
    % Draw series plot
    \addplot[no markers,blue,solid] coordinates {
        (-30.0, 72.2936809445)
        (-30.0, 60.0)
        (-30.0, 51.2077636525)
        (-30.0, 43.853778612)
        (-30.0, 37.3165236011)
        (-30.0, 31.3064462487)
        (-30.0, 25.6589062733)
        (-30.0, 20.2679010557)
        (-30.0, 15.0586473737)
        (-30.0, 9.9742217944)
        (-30.0, 4.96818353607)
        (0.0, 0.0)
        (150.0, 4.96818353607)
        (150.0, 9.9742217944)
        (150.0, 15.0586473737)
        (150.0, 20.2679010557)
        (150.0, 25.6589062733)
        (150.0, 31.3064462487)
        (150.0, 37.3165236011)
        (150.0, 43.853778612)
        (150.0, 51.2077636525)
        (150.0, 60.0)
        (150.0, 72.2936809445)
    };

    
    % Draw series plot
    \addplot[no markers,blue,solid] coordinates {
        (-12.5198297972, 87.1340160174)
        (-10.8933946491, 66.4218215218)
        (-8.94827556463, 56.6329870308)
        (-6.58677555363, 49.0239231237)
        (-3.67049650832, 42.5619875652)
        (0.0, 36.8698976458)
        (4.71500395395, 31.7883306171)
        (10.8933946491, 27.2747286804)
        (19.1066053509, 23.3821964255)
        (30.0, 20.2679010557)
        (43.897886248, 18.1948723388)
        (60.0, 17.4576031237)
        (76.102113752, 18.1948723388)
        (90.0, 20.2679010557)
        (100.893394649, 23.3821964255)
        (109.106605351, 27.2747286804)
        (115.284996046, 31.7883306171)
        (120.0, 36.8698976458)
        (123.670496508, 42.5619875652)
        (126.586775554, 49.0239231237)
        (128.948275565, 56.6329870308)
        (130.893394649, 66.4218215218)
        (132.519829797, 87.1340160174)
    };

    
    % Draw series plot
    \addplot[no markers,blue,solid] coordinates {
        (12.5198297972, 87.1340160174)
        (16.102113752, 69.5126848853)
        (20.1735701842, 61.5123318158)
        (24.7912808971, 55.7037913519)
        (30.0, 51.2077636525)
        (35.8175256444, 47.7256599062)
        (42.2163488397, 45.1432400456)
        (49.1066053509, 43.423570598)
        (56.3295034917, 42.5619875652)
        (63.6704965083, 42.5619875652)
        (70.8933946491, 43.423570598)
        (77.7836511603, 45.1432400456)
        (84.1824743556, 47.7256599062)
        (90.0, 51.2077636525)
        (95.2087191029, 55.7037913519)
        (99.8264298158, 61.5123318158)
        (103.897886248, 69.5126848853)
        (107.480170203, 87.1340160174)
    };










    \end{polaraxis}
\end{tikzpicture}
\end{sansmath}
    \caption{\captitle{Discrete direction reconstructions.} The possible
             direction reconstructions for a station with 3 detectors in
             a \SI{10}{\meter} equilateral triangle. The discrete
             solutions are due to the \SI{2.5}{\nano\second} sampling
             rate of the \adcs.}
    \label{fig:discrete_directions}
\end{figure}



\subsection{Order}

When only data is available a simply flat-shower from is fitted to the
data. Core information can be used when constructing a curved front,
because the center is known. Energy information can be used to get an
idea of the front curvature.


\section{Core reconstruction}

Once direction is reconstructed we can do core reconstruction..


\section{Energy reconstructions}

Reconstructions using LDF.. etc..
