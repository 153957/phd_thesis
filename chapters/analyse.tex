\chapter{Analyse}
\label{ch:analyse}


\section{Measured data}

Raw data consists of events with traces. An event is when the trigger
conditions are met in the \hisparc box of a station. A trigger requires
2 low signals in a 2-detector station or 2 high or 3 low signals in a
4-detector station. When a trigger occurs all \pmt channels of that
station are read-out, even if they do not have significant signals.
These traces are then processed by the \hisparcdaq/\pysparc. Derived
values, such as the baseline, standard deviation, number of peaks,
pulseheight and pulseintegral are determined for each trace. The traces
are also reduced by cutting parts from the front and end until the first
significant signal in any detector, with some buffer room?

During the trigger the \gps and \SI{400}{\mega\hertz} clock are used to
determine the \gps timestamp of the event. Besides this event data there
is also static data.

When a station is deployed, moved or changed a \gps self-survey is run,
this accurately determines the position of the \gps antenna. The
detector positions relative to the \gps antenna are also measured. If
they are moved the positions are again measured.

Finally the settings, e.g. \pmt voltages, trigger settings and other
settings, of a station are stored. Currently only when they change and
the button is pressed... To be sent every hour..

Config should be used for esd....


\section{Reconstruction chain}

Raw data is received by the frome server and stored in the \hisparc
datastore. This is done by the datastore software. Multiple listeners
receive data from stations and place that data in a temporary storage
location. A single writer process checks the temporary storage for new
data and writes it to the raw datastorage.

Every night all data from the previous day is processed by the pique
server and stored in the Event Summary Data. This data processing
consists of several steps.

First all events are sorted by their timestamp.

Then the duplicate events are removed. A station number and
extended timestamp (timestamp in nanoseconds) should be a unique
combination. Due to server crashes and errors in uploading duplicate
events may exist.

Next the MPV of the MIP peak for each detector is determined by fitting
the pulse integral histogram. Using this the number of MIPs in each
event is derived by dividing the pulse integral by the MPV. The MPV can
change when the voltages on the PMTs change or when the temperature of
the PMT changes. Higher (voltage/temp) -> higher (MPV)

After that the traces of each event are analysed to determine the moment
of the trigger and the arrival time (first pulse) in each detector.

In the ESD the sorted events with duplicates removed and MPV determined,
MIP derived, trigger reconstructed and arrival times determined data is
stored.

Station data is stored with the same structure as the raw storage.

When data for all stations is processed the next step is to see if links
exist between events. So coincidences between all stations are searched.
A coincidence is more than one event detected in a small time window by
multiple stations.

speed of light, distance, coincidence window, clusters, up time..

After this some histograms are made from the data and made available on
\url{http://data.hisparc.nl} to get an overview of the data.

Diagram different steps in reconstruction/checks.


\section{Quality assurance}


Why low quality? technical and/or software issues, or ..?
- Determine from number of events
- MIP peak location
- Expected number of coincidences
- Correctness of configuration
- Changes in configuration
etc...


\section{Data cuts}

Known detector positions..


\section{Direction reconstruction}

Before core reconstruction.
3 detection points can be solved analytical for flat front
More than 3 can be done with fit/regression..


\section{Core reconstruction}




\section{Energy reconstructions}

