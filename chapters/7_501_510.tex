\chapter{Interlaced detection stations}
\label{ch:501-510}


\section{Station 501 and 510}

At \nikhef is the longest operating four detector station, station 501. Since it started. In 2012? the detectors of station 501 were rearranged into a diamond formation, and simultaneously a new station, station 510, was placed offset \SI{2}{\meter} from 501. This configuration of two stations with overlapping area (not overlapping detectors), provides excellent opportunities to compare the efficiency and performance of two individual stations.

With the detectors of the two stations so close together the likelihood of detecting the same air showers, even for low energy showers, is fairly high. Still each station will often trigger for showers which do not cause a corresponding trigger in the other station.

[Efficiency of detecting showers of specific energy and distance..]


\section{Dataset overview}

There are xx days on which both stations had good data quality. This results in the following number of events:

[Table]

- events
- events (n >= 3)
- reconstructed events
- coincidences
- coincidences reconstructed in both


\section{Recorded particle density}


\section{Reconstructed directions}

Without any further filtering the angle between the reconstructed directions by the two stations looks like..


However, when only events in which both stations have higher particle densities are selected the reconstructions agree more and the difference is less.

By selecting events which higher densities only events which are either of higher energy or have the shower core closer to the center of the station are selected.


\section{}

\section{}
