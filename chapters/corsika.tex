\chapter{CORSIKA}
\label{ch:corsika}

\section{Why simulations?}

..


\section{Extensive air shower models}


Aires, CORSIKA or ...
Why was \corsika chosen.

It is based on various models for high/low energy hardon/electoweak
interactions; QGSJET/gheisha etc... Updated for LHC..


\section{Running simulations}

Simulations for various starting parameters were run. The combinations
of two seeds that can be given in the input where chosen to be unique
for each simulated shower, regardless of other parameters.


\subsection{Stoomboot}

In order to run a significant number of simulations the local Nikhef
computer cluster 'Stoomboot' was utilized. This cluster has around 300
cpus available, but uses a fair-use policy to give each group at Nikhef
equal computation time.

Simulation time.. \SI{10e17}{\electronvolt} is still feasible, taking
around 60 hours to complete. Lower energy showers take far less time, so
a large sample can easily be generated.


\subsection{Simulations catalogue}

Show in tables how many showers of each energy/primary/zenith we have.


\section{Simulations on clusters}

Detector/trigger/response simulations. Do we understand what we see?
