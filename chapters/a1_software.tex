\chapter{Software}
\label{ch:software}

\section{Structure}


\section{Distributed version control}

In February 2012 the \hisparc codebase was transitioned from Bazaar \cite{bazaar} to \git \cite{git} as distributed version control systems (DVCS). \git was chosen for its popularity and compatibility with \github \cite{github}. When Bazaar was still used the code repositories were hosted on a \nikhef server, this made it difficult for others to access the code and thereby harder to participate or contribute. On \github the software has been open sourced and can be downloaded easily. \github is now used to host all \hisparc code and teaching material repositories. \github provides a web-interface to manage projects, keep track of issues, and handle pull requests, which are proposals to adopt code changes. \github also provides a Graphical User Interface (GUI) for Mac and Windows called GitHub Desktop (cite github desktop), making it easy for those unfamiliar with the command line to use \git. Moreover, \git has a much larger userbase than other DVCS (cite DVCS statistics), which often makes it easier to find solutions to problems, because it is more likely that someone else has experienced the same problem.

\github also provides webpage hosting of static pages for repositories, called \github Pages. This feature is used to host software documentation and the compiled version of the teaching materials, for which the sources are also on \github.

For each project with documentation hosted by \github Pages contains a Makefile which streamlines the steps to generate an up to date version of the compiled documentation or teaching materials for the most recent commit. In some cases the Makefile also runs some commands to generate documents. For code repositories Sphinx (cite Sphinx) is used to make code documentation. For the LaTeX repositories, those with teaching materials, the LaTeX files are compiled to the common Portable Document Format (PDF) using the latest version of TeXLive.


\section{Programming language}

Given the outreach and open source character of \hisparc, programming languages were chosen that are relatively simple to learn, commonly used, easy to install, and readability of the code.

Most code repositories use \python as main language.
\sapphire, station-software, publicdb, datastore.

The web-based applications use \javascript. Which is the commonly used to make dynamic websites.

The proprietary programming language \labview (also VHDL?) is used for the  \daq software which runs on the detection stations. \labview is not version control friendly. Moreover, its files require paid software to read and compile, and the readability is lacking. These limitations makes it difficult to test the code, collaborate and switch maintainer of the code.


[TODO: Add some flowcharts]


\section{\hisparc Public Database}

\subsection{\api}

When making interactive applications (see \jsparc) there was a need for easy access all kinds of data and metadata in the Public Database. To facilitate this an \api has been added to the Public Database. The \api consists of a number of urls that return data in the form of a JavaScript Object Notation (\json) \cite{rfc7159}. A \json is a simple data format for the exchange of data. It is plain text which has been formatted to define objects, array, strings, numbers, booleans, and more. These are common data types in most programming languages and can therefore be interpreted by most languages and converted to a built-in data type.

The \api provides up-to-date access to the organizational structure of the \hisparc network of stations, and for individual stations data like the \gps location, detector placement, configurations, event traces, pulse height fits, and event counts can be retrieved.

A list of all possible urls can be found at \url{http://data.hisparc.nl/api/}. For example a list (array) of \hisparc stations, where each element contains the name and number of a station in a dictionary (object), can be retrieved via \url{http://data.hisparc.nl/api/stations/}, the result is:

\begin{minted}{json}
[
    {"name": "St. Nicolaaslyceum",
     "number": 2},
    {"name": "Het Amsterdams Lyceum",
     "number": 3},
    {"name": "Chr. Sch. Gem. Buitenveldert",
     "number": 5},
    {"name": "Bern. Nieuwetijt Coll. (Damstede)",
     "number": 6},
    {"name": "Joke Smit College (ROC)",
     "number": 7},
    …
    {"name": "Aarhus University 3",
     "number": 20003}
]
\end{minted}

Functions have been added to \sapphire and \jsparc to access the \api. These take care of constructing the urls, retrieving the result and converting the \json to a sensible format for \python and \javascript respectively.


\subsection{Event Summary Database}

Each morning at 4:00 \utc the Public database server starts to process new data, what processing entails is described in (secref other section). It looks for all raw data files that have been changed since the last processing. After processing the raw data the results are stored in the Event Summary Database (ESD). The ESD contains processed event and weather data, and coincidence data for the events. The ESD is then used to generate daily overviews per station and for coincidences.


\subsection{Data downloads}

The Public Database also provides access to the data in the ESD. This can be downloaded in the tab-separated-values format (\tsv) \cite{ianatexttsv}. There is no excesive \tsv specification, but there are generally agreed upon rules. The data is in plain text where each row is a seperate record. The values of each record are separated by a delimiter, in this case a tab character. Another similar and common format are \csv files which generally use a comma (,) as delimiter. However, because most of the \hisparc participants are in the Netherlands where the comma is usually the decimal delimiter some software might misinterpret the data. There are separate urls for downloading events, weather, and coincidences. The Public Database has web forms to easily download specific data. Both \sapphire and \jsparc have functions to access this data. \sapphire will automatically store the data in the \hdf format, which is also used by other parts of \sapphire.


\section{\sapphire}

\subsection{\pypi}

\sapphire is available via the Python Package Index (\pypi). This allows
package managers such as pip to easily find it. Ideally this means that installing \sapphire is as simple as running the following command in the command line:
%
\begin{minted}{bash}
   pip install hisparc-sapphire
\end{minted}

During the installation pip will also try to install other packages which are required for \sapphire. However, some of these packages require non-\python libraries, such as \hdf which is required for PyTables, pip can not install this because it is not a Python package. It might be easier to install the requirements before installing \sapphire itself.


\subsection{Clusters}

Using the \api \sapphire has access to the \gps coordinates of all stations. For stations that have submitted their detector positions, those are available as well. This makes it easy to setup a cluster with any selection of \hisparc stations. This can be used for simulations. Imaginary stations can be added to plan future station locations.


\subsection{\corsika}

\sapphire contains a module specifically for handling \corsika. This includes running the \corsika simulations, converting the results to \hdf, and creating an overview of the existing \corsika simulated showers.

The class to read the \corsika \fortran formatted files was originally created by Carlos? from the Auger Collaboration. It has been adapted for use in \hisparc. The files are converted to \hdf, because it is more easily queried to retrieve specific particles from the file. The \hdf files are optimized by creating a completely sorted index for the particle x coordinates. After this the data in the file is sorted on disk using this index. PyTables reads data in chunks, if data is sorted less chunks need to be read to get particles which fall in a specific range of coordinates. Moreover, the chunks are likely grouped together on disk, making it quicker to read them.


\subsection{Simulations}

The simulations accept input from \corsika ground particle data, LDFs
(density), or shower front models (timing). Detector and station simulations based on experiments and models provide realistic output data. The data can be analysed the same way as real \hisparc data (except that the input is
also known).

\begin{figure}
    \centering
    %\documentclass[a4paper,landscape,11pt]{article}
%
%\usepackage[a4paper]{geometry}
%\usepackage{tikz}
%\usetikzlibrary{arrows,pgfplots.groupplots,external}
%\usepackage{pgfplots}
%\pgfplotsset{compat=1.3}
%\usepgfplotslibrary{polar}
%\usepackage[detect-family]{siunitx}
%\usepackage[eulergreek]{sansmath}
%\usepackage{shape-datastore}
%
%\sisetup{text-sf=\sansmath}
%\usepackage{relsize}
%
%\pagestyle{empty}
%
%\begin{document}

\begin{tikzpicture}
[font=\sffamily,
 every matrix/.style={ampersand replacement=\&,column sep=1.75cm,row sep=1cm},
 source/.style={draw,thick,rounded corners,fill=yellow!20,inner sep=.3cm},
 process/.style={draw,thick,circle,fill=blue!20},
 sink/.style={source,fill=green!20},
 datastore/.style={draw,very thick,shape=datastore,inner sep=.3cm},
 dots/.style={gray,scale=2},
 fromto/.style={stealth'-stealth',shorten >=1pt,shorten <=1pt,semithick,font=\sffamily\footnotesize},
 to/.style={-stealth',shorten >=1pt,semithick,font=\sffamily\footnotesize},
 every node/.style={align=center},
]
\matrix{
    \& \node[source] (corsika) {CORSIKA}; \& \\
    \node[source] (ldf) {Lateral density\\function}; \& \node[datastore] (corsikadata) {CORSIKA\\simulations};  \& \node[source] (showerfront) {Shower front\\model}; \\
    \node[process] (detectorresponse) {Detector\\response}; \& \node[process] (simulations) {Simulations}; \& \node[datastore] (simulateddata) {Simulated\\data}; \\
    \node[source] (station) {HiSPARC stations}; \& \& \node[process] (reconstructions) {Reconstructions}; \& \node[datastore] (reconstructed) {Reconstructed\\data}; \\
    \node[datastore] (rawdata) {Raw data}; \& \node[process] (processevents) {Process\\events}; \& \node[datastore] (esd) {Event summary\\data}; \\
};

\draw[to] (corsika) -- node[midway,right] {pre-generate\\showers} (corsikadata);
\draw[to] (corsikadata) to [bend right=5] node[midway,left] {particles} (simulations);
\draw[to] (simulations) to [bend right=5] node[midway,right] {select\\shower} (corsikadata);
\draw[to] (ldf) -- node[midway,below,sloped] {particle density} (simulations);
\draw[to] (showerfront) -- node[midway,below,sloped] {front shape} (simulations);
\draw[fromto] (detectorresponse) -- (simulations);
\draw[to] (simulations) -- (simulateddata);
\draw[to] (station) -- (rawdata);
\draw[to] (rawdata) -- (processevents);
\draw[to] (processevents) -- (esd);
\draw[to] (simulateddata) -- (reconstructions);
\draw[to] (esd) -- (reconstructions);
\draw[to] (reconstructions) -- (reconstructed);

\end{tikzpicture}

%\end{document}

    \caption{\captitle{\sapphire data chain.} The flow from
             creating/detecting data to reconstructed data.}
    \label{fig:sapphire-flow}
\end{figure}


\subsection{Data quality}

This code flags bad data and ensures that the data going into the reconstruction is good data.

[TODO: Write code to determine data quality.]


\section{Code testing}

Tests have been written to ensure that programming code does what is expected, even when it is modified later. Code tests can be very specific and precisely check the output given some inputs, check how often another function is called, and ensure that the correct errors are thrown when given invalid input. More general tests are also used to validate code validity and coding style. A lot of effort has been put into ensuring that all \python code adheres to the same coding style, the Python Enhancement Proposal (PEP) 8 - Style Guide for Python Code. Using a consistent style makes code easier to read and makes it easier to find bugs.

These tests can be run locally while developing code. However, developers might forget to run tests or do not want to spend time testing. To ensure that new code is always tested the \sapphire and \jsparc repositories have the \travis service \cite{travis} added. After new commits are pushed to those repository \travis starts a Virtual Machine that will download the repository and install any requirements we have defined and run the tests. The results are then displayed on the repository page, and the person that pushed the code will get an e-mail if the tests fail. For \sapphire commits must pass the tests before they can be pushed to the master branch.


\subsection{Code coverage}

Having a test for a function does not guarantee that every bit of code is tested. To check which parts of the code are tested we use a code coverage tool. While \travis runs tests it keeps track of which lines of code in \sapphire are run. At the end of the tests a report is sent to \coveralls \cite{coveralls} and \codecov \cite{codecov} which then shows overviews of which lines have been tested and which have not. However, testing each line once does not ensure proper coverage. A common scenario is that only one path through an if statement is checked, and since else is optional in \python it is not always obvious that the alternative is not tested. \codecov takes this into account and reports on missed execution branches.


\subsection{Vagrant}

The Public Database is more complex than \sapphire and \jsparc. It requires the Django package \cite{django}, a database and raw data to be fully tested. These tests can be very time consuming. When developing code for the Public Database, it is best to test it in en environment similar to the live server. To accomplish this we use a local virtual machine that is provisioned by scripts to install the required packages and setup some configurations. These scripts for provisioning a test machine can also be used to setup the live server, this ensures that the test and live server are almost identical.

[TODO: Testing publicdb]


\section{Servers}


[flowchart of data flow between servers/users/stations]

neckar - to be removed? if we host a website on GitHub
pique - public database, vagrant
frome - datastore
tietar - vpn nagios - needs attention
trave - storage
