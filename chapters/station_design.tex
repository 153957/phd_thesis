\section{Station design}

\subsection{HiSPARC electronics}

The \hisparc hardware (versions II and III) combine \pmt control, signal
readout/daq, triggering logic, and the \gps receiver into one unit. The
unit can be controlled and readout from a PC connected via USB
\cite{messages}. The original version of the \hisparc hardware did not
include the \daq and \gps receiver, those were separate modules and it
could not be controlled via USB. Since that original version is no
longer used by any station in the network it is ignored in the following
discussion.

Master/Slave - 4 channels

PMT control

Readout pos/neg

Triggering logic

Master and Slave contain the same firmware, each should be configured
with the same settings. The boxes are connected with two
\SI{0.3}{\meter} UTP cables that facilitate the communication between
the two units. The Master sends its PPS data to the Slave, this allows
the Slave to keep track of its own CTP (clock ticks between PPS) values.
Both units send One Second Messages to the PC, each with their own CTP
values. All \adc values are checked against the thresholds at each clock
tick (\SI{200}{\mega\hertz}). Each channel has two ADCs, one checks the
signal on the positive flank of the clock, the other on the negative.
The results of the comparison to the thresholds is also communicated
between the Master and Slave, because they need to know what the
conditions in all four channels is to determine if a trigger occurred.
When a unit determines a trigger takes place it will latch the number of
clock ticks since the last CTP, this is called the CTD. The event will
also contain the timestamp of the last PPS. Then each unit will send
their own measured data message to the pc. The PC will need to figure
out which Master and Slave events belong together. Using CTP value from
the one second message at the end of the second in which the event
occurred. This is used to determine at what fraction of the second the
event occurred. Further corrected by taking the Quantization errors into
account. The quantization error for each PPS is in the following One
second message. So the One second message at the end of the second after
the event is also required for the quantization error of the end value
of the event second.


PPS/Events




Readout electronics. Two channels, each with two ADCs which readout a
signal up to \SI{-2}{\volt}. And return a 12-bit value (4096 ADCcounts
range). Each ADC is sampled at 200 Mhz, using two ADCs per channel gives
400 Mhz (2.5 ns) per channel. Nikhef designed. Some (derivatives) used
in other experiments.

Multiple versions, Latest version \hisparciii. New software needed,
remote firmware updates.

GPS receiver module \cite{ref:trimble}

Multiple detectors connected to \hisparc electronics makes a station.

\begin{figure}
    \centering
    \input{plots/hisparc-experiment/4_detector_station}
    \caption{\captitle{Standard 4-detector station layout.} The colored
             rectangles represent the scintillator detectors.}
    \label{fig:4_detector_station}
\end{figure}
