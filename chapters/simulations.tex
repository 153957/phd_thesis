\chapter{Simulations}
\label{ch:simulations}

\section{Why simulations?}

Simulations are used to check if the data we see is what we expected.



\section{Extensive air shower models}

Aires, CORSIKA or ...
Why was \corsika chosen.

It is based on various models for high/low energy hardon/electoweak
interactions; QGSJET/gheisha etc... Updated for LHC..


\subsection{Simulations catalogue}

Show in tables how many showers of each energy/primary/zenith we have.


\section{Detector simulation}

trigger efficiency
simulatiosn
every step eplained and understood
triggere representateren


Simulations for various starting parameters were run. The combinations
of two seeds that can be given in the input where chosen to be unique
for each simulated shower, regardless of other parameters.


\subsection{Stoomboot}

In order to run a significant number of simulations the local Nikhef
computer cluster 'Stoomboot' was utilized. This cluster has over 300
cpus available and uses a fair-use policy to give each group at Nikhef
equal computation time. 

Simulation time for each simulation is different because the number of
particles in a simulations will be different each time. Stoomboot has a
maximum job time of \SI{72.96}{\hour}. This allows for
\SI{10e17}{\electronvolt} showers, which take around 60 hours to
complete. Lower energy showers take far less time (energy proportional
to time?), so large sample of showers can easily be generated.


\section{Simulations on clusters}

We have some simple simulations for flat and curved shower fronts.
Does not include particle density.
Detector/trigger/response simulations. Do we understand what we see?


\section{Realistic simulation}

Simulation with fluxes and azimuth/zenith distributions..
etc..
Acceptance; angle, energy and particles.


\section{Analysis}

\todo{Information about shower from simulations}

Timing information, front shape, timing distribution, particle arrival times..
Air pressure dependence.

Particle density, effect of inclination..
