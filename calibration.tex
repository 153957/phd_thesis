\chapter{Calibration}
\label{ch:calibration}

\section{\pmt}

Pulseheight/integral
Number of particles, leptons and gammas, separate?


\section{\gps accuracy}

This described time difference calibration measurements performed on
HiSPARC II and III electronic boxes. This started after Dr. David
Fokkema discovered an unexpected time offset between timestamps of
events in a test where two HiSPARC stations (501 and 502) were triggered
with the same pulse generator. Since these stations are very close
($\sim\SI{100}{\meter}$) a minimal time offset was expected, any offset
or large standard deviation was expected to be the result of errors in
\gps accuracy and position. However, a large offset ($\Delta t
\sim\SI{40}{\nano\second}$) was found. Further tests with other HiSPARC
boxes and various setups were performed. Found was that both the HiSPARC
boxes and GPS antennas have different offsets contributing to time
differences in the resulting time measurements. The offsets in the
HiSPARC boxes can be as high as $|\Delta t| = \SI{50}{\nano\second}$ and
in the GPSs $|\Delta t| \approx \SI{8}{\nano\second}$. While the
$\sigma_{\Delta t}$ remains within GPS specifications ($\sigma_{\Delta
t} < \SI{4.5}{\nano\second}$).


\subsection{Calibration}

[tijdtest]
One reference, compared to another \hisparc electronics box.
Each connected to a/same gps, triggered by same pulse generator.
\gps timestamps compared.

\subsection{Offsets between \hisparc electronics}

Found different offsets between boxes.
Constant over time.
Can be corrected.

Importance of 24 hour self-survey.
