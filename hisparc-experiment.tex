\chapter{HiSPARC Experiment}
\label{ch:hisparc-experiment}

Detecting cosmic-ray extensive air showers with affordable detectors.

Large area, sparse array.


\section{Network status}

There are, as of this writing, 120 \hisaprc stations in the world. Most
are located on high-schools. And most are located in the Netherlands.

[plot for number of active stations from 2004-ish untill now]


\begin{figure}
    \centering
    \includegraphics[width=0.7\linewidth]{plots/network/number_of_stations_with_data_per_day}
    \caption{Number of active stations per day from January 1st 2004 until now.}
    \label{fig:number_of_stations_with_data_per_day}
\end{figure}


\subsection{International}

In Germany one of our stations (70001) was placed inside the \kascade
experiment in Karlsruhe on July 1, 2008. This was done to calibrate and
test the direction reconstruction accuracy of a single \hisaprc station.
Although the \kascade-GRANDE experiment officially shut down on March
30, 2009, it was kept active until November, 26th 2011 as a test
facility for some test setups, including our detector. The \kascade
experiment triggered the \hisparc station when it detected a shower. It
was triggered more than \num{9e7} (in publicdb, more on external hdd?
and in `/databases/kascade`?) times in this period. The detectors have
been repurposed and are now part of station 508 on the Science Park.

Since 2007 there has been a \hisparc staiton in
Denmark. There are currently 3 operational stations at the Aarhus
University. These stations are managed by Uffe Amelung Fredens.
Fredens is writing student materials for Danish high school students.

In Bristol, England a new cluster started to form around Bristol in
March 2012. This was initiated by Dr. Jaap Velthuis who is the cluster
coordinator for Bristol. Since then schools in Bristol, Bath, Swindon
and Chippenham have joined the network. More on the way..

[Finland perhaps?]


\section{Detector design}

\subsection{Photo multiplier}

Nikhef designed high-voltage voltage-divier circuit.
Cockcroft-Walton voltage multiplier circuit.
High linearity?
